\begin{abstract}
%225 out of the maximum 225 words are used
Benchmark surveillance \textit{tests} for diagnosing disease \textit{progression} (biopsies, endoscopies, etc.) in early-stage chronic non-communicable disease patients (e.g.,~cancer, lung diseases) are usually invasive. For detecting progression timely, over their lifetime, patients undergo numerous invasive tests planned in a fixed one-size-fits-all manner (e.g.,~biannually). We present personalized test schedules based on progression-risk, that aim to optimize the number of tests (burden) and time delay in detecting progression (shorter is beneficial) better than fixed schedules. Our motivation comes from the problem of scheduling biopsies in prostate cancer surveillance studies.

Using joint models for time-to-event and longitudinal data, we consolidate patients' longitudinal data (e.g.,~biomarkers) and results of previous tests, into individualized future cumulative-risk of progression. We then create personalized schedules by planning tests on future visits where the predicted cumulative-risk is above a particular \textit{threshold} (e.g.,~5\% risk). This schedule is updated on each follow-up with newly gathered data. To find the optimal risk threshold, we minimize a utility function of the expected number of tests (burden) and expected time delay in detecting progression (shorter is beneficial) for different thresholds. We estimate these two quantities in a patient-specific manner for following any schedule by utilizing the predicted risk profile of the patient. Patients/doctors can employ these quantities to compare various personalized and fixed schedules objectively. Last, we implement our methodology in a web-application for prostate cancer patients.
\end{abstract}