% !TEX root =  ../main_manuscript.tex 
\section{Patients and Methods}
\subsection{Study Cohort}
\label{subsec:cohort}
For developing a statistical model to predict upgrading-risk, we used the world's largest AS dataset, Prostate Cancer International Active Surveillance or PRIAS~\citep{bul2013active}, dated April 2019 (Table~\ref{table:prias_summary}). In PRIAS, biopsies were scheduled at year one, four, seven, ten, and additional yearly biopsies were scheduled when PSA doubling time was between zero and ten years. We selected all 7813~patients who had Gleason grade group~1 at inclusion in AS. Our primary event of interest is an increase in this Gleason grade group observed upon repeat biopsy, called \textit{upgrading} (1134~patients). Upgrading is a trigger for treatment advice in PRIAS. Some examples of treatment options in active surveillance are radical prostatectomy, brachytherapy, definitive radiation therapy, and other alternative local treatments such as cryosurgery, High Intensity Focused Ultrasound, and External Beam Radiation Therapy. Comprehensive details on treatment options and their side effects are available in EAU-ESTRO-SIOG guidelines on prostate cancer~\citep{mottet2017eau}. In PRIAS 2250 patients were provided treatment based on their PSA, the number of biopsy cores with cancer, or anxiety/other reasons. However, our reasons for focusing solely on upgrading are that upgrading is strongly associated with cancer-related outcomes, and other treatment triggers vary between cohorts~\citep{nieboer2018active}.

For externally validating our model's predictions, we selected the following largest (by the number of repeated measurements) six cohorts from Movember Foundation's GAP3 database~\citep{gap3_2018} version~3.1, covering nearly 73\% of the GAP3 patients: the University of Toronto AS (Toronto), Johns Hopkins AS (Hopkins), Memorial Sloan Kettering Cancer Center AS (MSKCC), King's College London AS (KCL), Michigan Urological Surgery Improvement Collaborative AS (MUSIC), and University of California San Francisco AS (UCSF, version~3.2). Only patients with a Gleason grade group~1 at the time of inclusion in these cohorts were selected. Summary statistics are presented in Supplementary~A.2.

\begin{table}
\small\sf\centering
\caption{\textbf{Summary of the PRIAS dataset as of April 2019}. The primary event of interest is upgrading, that is, increase in Gleason grade group from group~1~\citep{epsteinGG2014} to 2 or higher. IQR:~interquartile range, PSA:~prostate-specific antigen. Study protocol URL: \url{https://www.prias-project.org}}
\label{table:prias_summary}
\begin{tabular}{lr}
\toprule
\textbf{Characteristic} & \textbf{Value}\\
\midrule
%Total centers & $> 100$\\
Total patients & 7813\\
Upgrading (primary event) & 1134\\
Treatment & 2250\\
Watchful waiting & 334\\
Loss to follow-up & 249\\
Death (unrelated to prostate cancer) & 95\\
Death (related to prostate cancer) & 2\\
\midrule
Median age at diagnosis (years) & 66 (IQR: 61--71)\\
Median maximum follow-up per patient (years) &  1.8 (IQR: 0.9--4.0)\\
Total PSA measurements & 67578\\
Median number of PSA measurements per patient &  6 (IQR: 4--12)\\
Median PSA value (ng/mL) & 5.7 (IQR: 4.1--7.7)\\
Total biopsies & 15686\\
Median number of biopsies per patient &  2 (IQR: 1--2)\\
\bottomrule
\end{tabular}
\end{table}

\paragraph{Choice of predictors:} In our model, we used all repeated PSA measurements, the timing of the previous biopsy and Gleason grade, and age at inclusion in AS. Other predictors such as prostate volume, MRI results can also be important. MRI is utilized already for targeting biopsies, but regarding its use in deciding the time of biopsies, there are arguments both for and against it~\citep{kasivisvanathan2020magnetic,chesnut2019role,schoots2015magnetic}. MRI is still a recent addition in most AS protocols. Consequently, repeated MRI data is very sparsely available in both PRIAS and GAP3 databases to make a stable prediction model. Prostate volume data is also sparsely available, especially in validation cohorts. Based on these reasons, we did not include them in our model.