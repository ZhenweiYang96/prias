% !TEX root =  ../main_manuscript.tex 
\begin{abstract}
\texttt{Objective}: To develop a model and methodology for predicting the risk of Gleason \emph{upgrading} in prostate cancer active surveillance (AS) patients, and using the predicted risks to create risk-based \emph{personalized} biopsy schedules as an alternative to one-size-fits-all schedules (e.g., annually). Furthermore, to assist patients and doctors in making shared decisions of biopsy schedules, by providing them quantitative estimates of the \emph{burden} and \emph{benefit} of opting for personalized versus any other schedule in AS. Last, to externally validate our model and implement it along with personalized schedules in a ready to use web-application.\\

\texttt{Materials and Methods}: We used longitudinal prostate-specific antigen (PSA) measurements, timing and results of previous biopsies, and age at baseline from the world's largest AS study, Prostate Cancer Research International Active Surveillance or PRIAS (7813 patients, 1134 experienced upgrading). We fitted a Bayesian joint model for time-to-event and longitudinal data to the PRIAS dataset. We then externally validated our model in the largest six AS cohorts of the Movember Foundation's Global Action Plan (GAP3) database (${>20,000}$ patients, 27 centers worldwide), covering nearly 73\% of all GAP3 patients. We used the predicted upgrading-risks from the validated models to schedule biopsies whenever a patient's risk of upgrading was above a certain threshold. To assist patients in choice of this threshold to compare the resulting schedule with currently practiced schedules, we provided them the timing and the total number of biopsies (burden) planned, and the predicted time delay in detecting upgrading (shorter is better) for each schedule.\\

\texttt{Results}: The cause-specific cumulative upgrading-risk at year five of follow-up was 35\% in PRIAS, and at most 50\% in GAP3 cohorts. In the PRIAS based model, PSA velocity was a stronger predictor of upgrading (Hazard~Ratio:~2.47, 95\%CI:~1.93--2.99) than PSA value (Hazard~Ratio:~0.99, 95\%CI:~0.89--1.11). Our model had a moderate area under the receiver operating characteristic curve (0.6--0.7) in validation cohorts. The prediction error was moderate (0.1--0.2) in GAP3 cohorts where the impact of PSA value and velocity on upgrading-risk was similar to PRIAS, but large (0.2--0.3) otherwise. Our model required recalibration of baseline upgrading-risk in validation cohorts. We used predicted upgrading-risk from the validated model to create personalized biopsy schedules for real AS patients and implemented them in a web-application (\url{http://tiny.cc/biopsy}).\\

\texttt{Conclusions}: We successfully developed and validated a model for predicting upgrading-risk, and providing risk-based personalized biopsy decisions, in prostate cancer AS. Personalized prostate biopsies are a novel alternative to fixed one-size-fits-all schedules that may help to reduce unnecessary prostate biopsies while maintaining cancer control. The model and schedules made available via a web-application enable shared decision making of biopsy schedules by comparing fixed and personalized schedules on total biopsies and expected time delay in detecting upgrading.
\end{abstract}