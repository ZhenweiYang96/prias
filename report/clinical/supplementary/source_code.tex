% !TEX root =  ../supplementary.tex
\section{Source Code}
The R code for fitting the joint model to the PRIAS dataset, is at \url{https://github.com/anirudhtomer/prias/tree/master/src/clinical_gap3}. We refer to this location as `R\_HOME' in the rest of this document.

\subsection{Fitting the joint model to the PRIAS dataset}
\textbf{Accessing the dataset:}
The PRIAS dataset is not openly accessible. However, access to the database can be requested via the contact links at \url{www.prias-project.org}.\\

\textbf{Formatting the dataset:}
This dataset however is in the so-called wide format and also requires removal of incorrect entries. This can be done via the R script \url{R_HOME/dataset_cleaning.R}. This will lead to two R objects, namely `prias\_final.id' and `prias\_long\_final'. The `prias\_final.id' object contains information about time of GS7 for PRIAS patients. The `prias\_long\_final' object contains longitudinal PSA measurements, the time of biopsies and results of biopsies.\\

\textbf{Fitting the joint model:}
We use a joint model for time to event and longitudinal data to model the evolution of PSA measurements over time, and to simultaneously model their association with the risk of GS7. The R package we use for this purpose is called \textbf{JMbayes} (https://cran.r-project.org/web/packages/JMbayes/JMbayes.pdf). The API we use, however, are currently not hosted on CRAN, and can be found here:
\url{https://github.com/anirudhtomer/JMbayes}. The joint model can be fitted via the script \url{R_HOME/analysis.R}. It takes roughly 6 hours to run on an Intel core-i5 machine with 4 cores, and 8GB of RAM. 

The graphs presented in the main manuscript, and the supplementary material can be generated by the scripts \url{R_HOME/fittingModel/demographs.R}, and \url{R_HOME/fittingModel/modelDiagnostic.R}, respectively.

\subsection{Running the simulation study}
The simulation study can be run by the following script: \url{R_HOME/simulationStudy/controller.R}. Although it depends on other files, the code should run without errors as long as the directory structure is maintained. An entire simulation study may take weeks to run. However, this can be controlled via the variable `dataSetNums' in the script. Graphs related to simulation study results can be generated from the scripts \url{R_HOME/simulationStudy/decisionMakingGraph.R} and \url{R_HOME/simulationStudy/produceResults.R} 