% !TEX root =  ../main_manuscript.tex 
\section{Results}
For patients in the PRIAS dataset, the probability of obtaining reclassification within the first five and ten years is 33\% and 42\%, respectively (see Figure \ref{fig:npmle_plot}). That is, more than 50\% of the patients may not require any biopsy in the first ten years. We refer to them as \textit{slow progressing} patients hereafter. For every ten years increase in a patient age the corresponding adjusted hazard ratio of reclassification is 1.45~(95\%CI:~1.29--1.61). For an increase in fitted $\log_2\{\mbox{PSA + 1}\}$ value from the first quartile of fitted value (2.67) to the third quartile (2.82), the corresponding adjusted hazard ratio of reclassification is 1.00~(95\%CI:~0.98--1.02). On the other hand an increase in fitted $\log_2\{\mbox{PSA + 1}\}$ velocity from the first quartile of fitted velocity (-0.04) to the third quartile (0.15), the corresponding adjusted hazard ratio of reclassification is 2.45~(95\%CI:~1.83--2.95). These results indicate that the velocity of $\log_2\{\mbox{PSA + 1}\}$ measurements is a stronger predictor of hazard of reclassification than the $\log_2\{\mbox{PSA + 1}\}$ value.

The time dependent area under the receiver operating characteristic curves (AUC) and the root mean squared prediction error (RMSPE) with 95\% CI are shown in Figure. The results are comparable for internal and external validation in cohorts that are similar to the PRIAS cohort. That is, the model may also be useful for risk prediction in other cohorts such as the Toronto AS cohort. 

Using the fitted model we next predict the 

\begin{table}
\small\sf\centering
\caption{\textbf{Assessment of predictions of Gleason~$\geq$~7 (GS7) from our model using}, area under the receiver operating characteristic curve (AUC), and root mean squared prediction error (RMSPE).}
\label{table:prediction_validation}
\begin{tabular}{lr}
\hline
\hline
Characteristic & Value\\
\hline
Total patients & 7813\\
Gleason $\geq$ 7 (primary event) & 1134\\
Treatment & 2250\\
Watchful waiting & 334\\
Loss to follow-up & 250\\
Death (unrelated to prostate cancer) & 95\\
Death (related to prostate cancer) & 2\\
\hline
Median age at diagnosis (years) & 61 (IQR: 66--71)\\
Median follow-up period per patient (years) &  1.8 (IQR: 0.9--3.99)\\
Total PSA measurements & 65798\\
Median number of PSA measurements per patient &  6 (IQR: 4--12)\\
Median PSA value (ng/mL) & 5.7 (IQR: 4.1--7.7)\\
Total biopsies & 15563\\
Median number of biopsies per patient &  2 (IQR: 1--3)\\
\hline
\end{tabular}
\end{table}