% !TEX root =  ../main_manuscript.tex 
\section{Results}
For patients in the PRIAS dataset, probability of obtaining reclassification within the first five and ten years is 33\% and 42\%, respectively (see Figure \ref{fig:npmle_plot}). That more than 50\% of the patients may not require any biopsy in the first ten years. We refer to them as \textit{slow progressing} patients hereafter. For every ten years increase in a patient age the corresponding adjusted hazard ratio of reclassification is 1.45~(95\%CI:~1.29--1.61). For an increase in fitted $\log_2\{\mbox{PSA + 1}\}$ value from the first quartile of fitted value (2.67) to the third quartile (2.82), the corresponding adjusted hazard ratio of reclassification is 1.00~(95\%CI:~0.98--1.02). On the other hand an increase in fitted $\log_2\{\mbox{PSA + 1}\}$ velocity from the first quartile of fitted velocity (-0.04) to the third quartile (0.15), the corresponding adjusted hazard ratio of reclassification is 2.45~(95\%CI:~1.83--2.95). These results indicate that the velocity of $\log_2\{\mbox{PSA + 1}\}$ measurements is a stronger predictor of hazard of reclassification than the $\log_2\{\mbox{PSA + 1}\}$ value.

The area under the receiver operating characteristic curves and the prediction error over time are shown.

