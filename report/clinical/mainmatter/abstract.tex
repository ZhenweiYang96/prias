% !TEX root =  ../main_manuscript.tex 
\begin{abstract}
\texttt{Background}: Prostate cancer patients enrolled in active surveillance (AS) undergo repeat biopsies. When biopsy Gleason~grade~$\geq$~2 (reclassification), treatment is commonly advised. Many patients never experience reclassification, yet undergo biopsies frequently.\\

\texttt{Objective}: Better balance the number of biopsies and time delay in detection of reclassification.\\

\texttt{Design, Setting, and Participants}: World's largest AS study, PRIAS; 7813 patients, 1134 experienced reclassification; prostate-specific antigen (PSA) and repeat biopsy data available.\\

\texttt{Outcome Measurements, and Statistical Analysis}: Bayesian joint model based on accumulated clinical data used to predict patient-specific risk of reclassification. This risk was utilized to schedule personalized biopsies. Personalized and fixed schedules compared on number of biopsies and model estimated time delay in detection of reclassification for each schedule. Model validated externally in largest five AS cohorts of GAP3 database. Methodology implemented in a web-application.\\

\texttt{Results and Limitations}: Rate of reclassification in PRIAS was 35\% at year 5 of follow-up. PSA velocity stronger predictor of reclassification (Hazard~Ratio:~2.47, 95\%CI:~1.93--2.99), than PSA value (Hazard~Ratio:~0.99, 95\%CI:~0.89--1.11). Validation: Area under ROC curve for risk predictions between 0.55 and 0.75 for PRIAS, Johns Hopkins, Toronto, and Memorial Sloan Kettering AS cohorts. Model required recalibration for all external cohorts except Johns Hopkins cohort.\\

\texttt{Conclusions}: We used risk predictions of reclassification to schedule personalized biopsies for AS patients. To assist patients/doctors in shared decision making of appropriate biopsy schedule, we provided them expected time delay in detection of reclassification, for personalized/fixed schedules. Our model is externally validated, and our methodology is available for multiple AS cohorts as a web-application.\\

\texttt{Patient Summary}: Personalized biopsy schedules are a novel alternative to fixed biopsy schedules. They rely on patient-specific risk of reclassification and can offer better balance between number of biopsies and time delay in detection of reclassification than current schedules.
\end{abstract}