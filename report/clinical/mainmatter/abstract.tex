% !TEX root =  ../main_manuscript.tex 
\begin{abstract}
\texttt{Background}: Prostate cancer patients enrolled in active surveillance (AS) undergo repeat biopsies. When biopsy Gleason~grade~$\geq$~2 (reclassification), treatment is commonly advised. Many patients never experience reclassification, yet undergo biopsies frequently.\\

\texttt{Objective}: Better balance the number of biopsies and time delay in detection of reclassification.\\

\texttt{Design, Setting, and Participants}: World's largest AS study, PRIAS; 7813 patients, 1134 experienced reclassification; prostate-specific antigen (PSA) and repeat biopsy data available.\\

\texttt{Outcome Measurements, and Statistical Analysis}: Bayesian joint model based on accumulated clinical data used to predict patient-specific risk of reclassification. Predicted risks were utilized to schedule personalized biopsies. Personalized and fixed schedules were compared by number of biopsies, and model estimated expected time delay in detection of reclassification for each schedule. Model was validated externally in largest five AS cohorts from GAP3 database. Methodology implemented in a web-application.\\

\texttt{Results and Limitations}: Rate of reclassification in PRIAS was 50\% in 10 years of follow-up. PSA velocity was a stronger predictor of reclassification (Hazard~Ratio:~2.47, 95\%CI:~1.93--2.99), than PSA value (Hazard~Ratio:~0.99, 95\%CI:~0.89--1.11). Internal validation: Time varying area under ROC curve for reclassification prediction between 0.62 and 0.69, and prediction error between 0.23 and 0.37. External validation: Results similar to internal validation only for Toronto, Memorial Sloan Kettering, and Johns Hopkins AS cohorts.\\

\texttt{Conclusions}: We developed personalized biopsy schedules as alternative to fixed schedules for AS patients. To assist patients/doctors in biopsy decisions, we provided them expected time delay in detection of reclassification, for both personalized and fixed schedules. We implemented our methodology in a web-application.\\

\texttt{Patient Summary}: Personalized biopsy schedules are a novel alternative to fixed biopsy schedules (e.g., yearly biopsies). They rely on patient-specific risk of reclassification.  Personalized schedules can offer a better balance between number of biopsies and delay in detection of Gleason upgrade than yearly biopsies.
\end{abstract}