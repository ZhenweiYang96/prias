% !TEX root =  ../main_manuscript.tex 
\begin{abstract}
\texttt{Background}: Prostate cancer active surveillance (AS) patients undergo repeat biopsies. Active treatment is advised when biopsy Gleason grade group~$\geq$~2 (\textit{upgrading}). Many patients never experience upgrading, yet undergo biopsies frequently. Personalized biopsy decisions based on upgrading-risk may reduce patient burden.\\

\texttt{Objective}: Develop a risk prediction model and web-application to assist patients/doctors in personalized biopsy decisions.\\

\texttt{Design, Setting, and Participants}: Model development: world's largest AS study PRIAS, 7813 patients, 1134 experienced upgrading; External validation: largest six cohorts of Movember Foundation's GAP3 database (${>20,000}$ patients, 27 centers worldwide); Data: repeat prostate-specific antigen (PSA) and biopsy Gleason grade.\\

\texttt{Outcome Measurements, and Statistical Analysis}: A Bayesian joint model fitted to the PRIAS dataset. This model was validated in GAP3 cohorts using risk prediction error, calibration, area under ROC (AUC). Model and personalized biopsy schedules based on predicted risks were implemented in a web-application.\\

\texttt{Results and Limitations}: Cause-specific cumulative upgrading-risk at year five of follow-up: 35\% in PRIAS, at most 50\% in GAP3 cohorts. PRIAS based model: PSA velocity was a stronger predictor of upgrading (Hazard~Ratio:~2.47, 95\%CI:~1.93--2.99) than PSA value (Hazard~Ratio:~0.99, 95\%CI:~0.89--1.11). Validation: Moderate AUC (0.55--0.75) in PRIAS and GAP3 cohorts. Moderate prediction error (0.1--0.3) in GAP3 cohorts where impact of PSA value and velocity on upgrading-risk was similar to PRIAS, but large (0.3--0.45) otherwise. Recalibration advised for cohorts not validated in this work.\\

\texttt{Conclusions}: We successfully developed and validated a model for predicting upgrading-risk, and providing risk-based personalized biopsy decisions, in prostate cancer AS. The model made available via a web-application enables shared decision making of biopsy schedules by comparing fixed and personalized schedules on total biopsies and expected time delay in detecting upgrading.\\

\texttt{Patient Summary}: Personalized prostate biopsies are a novel alternative to fixed one-size-fits-all schedules. The underlying statistical models are made available through a user-friendly web-application and may help to reduce unnecessary prostate biopsies while maintaining cancer control.
\end{abstract}