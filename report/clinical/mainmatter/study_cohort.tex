% !TEX root =  ../main_manuscript.tex 
\section{Patients and Methods}

\subsection{Study Cohort}
\label{subsec:cohort}
Prostate Cancer International Active Surveillance (PRIAS) is an ongoing (December~2006~--~to date) prospective cohort study of men with low- and very-low risk prostate cancer diagnoses~\citep{bul2013active}. More than 100 medical centers from 17 countries contributed to PRIAS, using a common protocol (\url{www.prias-project.org}). Upon inclusion in AS, PSA was measured quarterly for the first two years of follow-up and semiannually thereafter. Biopsies were scheduled at year one, four, seven, and ten of follow-up. Additional yearly biopsies were scheduled when PSA doubling time was between zero and ten years.

We selected all 7813 patients who had Gleason grade~1~\citep{epsteinGG2014} at the time of inclusion in PRIAS (Table~\ref{table:prias_summary}). Our primary event of interest is increase in this Gleason grade upon repeat biopsy, called \textit{reclassification} (1134 patients). Reclassification is a trigger for treatment advice in PRIAS. Although, 2250 patients were provided treatment on the basis of their PSA, or number of biopsy cores with cancer, or anxiety/other reasons. Our reasons for focusing solely on reclassification are, namely reclassification is strongly associated with cancer related outcomes, and other triggers for treatment vary between cohorts.

\begin{table}
\small\sf\centering
\caption{\textbf{Summary of the PRIAS dataset}. The primary event of interest is reclassification, that is, increase in Gleason grade from grade~1 to 2 or higher. IQR:~interquartile range, PSA:~prostate-specific antigen.}
\label{table:prias_summary}
\begin{tabular}{lr}
\toprule
\textbf{Characteristic} & \textbf{Value}\\
\midrule
Total patients & 7813\\
Reclassification (primary event) & 1134\\
Treatment & 2250\\
Watchful waiting & 334\\
Loss to follow-up & 250\\
Death (unrelated to prostate cancer) & 95\\
Death (related to prostate cancer) & 2\\
\midrule
Median age at diagnosis (years) & 66 (IQR: 61--71)\\
Median follow-up period per patient (years) &  1.8 (IQR: 0.9--4.0)\\
Total PSA measurements & 67578\\
Median number of PSA measurements per patient &  6 (IQR: 4--12)\\
Median PSA value (ng/mL) & 5.7 (IQR: 4.1--7.7)\\
Total biopsies & 15686\\
Median number of biopsies per patient &  2 (IQR: 1--2)\\
\bottomrule
\end{tabular}
\end{table}