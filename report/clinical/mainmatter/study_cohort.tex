% !TEX root =  ../main_manuscript.tex 
\section{Patients and Methods}

\subsection{Study Cohort}
Prostate Cancer International Active Surveillance (PRIAS) is an ongoing prospective cohort study of men with low- and very-low risk PCa diagnoses \cite{bul2013active}. More than 100 medical centers from 17 countries contribute in PRIAS, using a common study protocol (\url{www.prias-project.org}). We used the data collected between December~2006 (beginning of PRIAS study) and May~2019. The PSA was measured every three months until year two of follow-up and every six months thereafter. Biopsy schedule was year one, four, seven, and ten, and additional yearly biopsies when PSA doubling time is between zero and ten years. The primary event of this work is Gleason~$\geq$~7 (GS7). It was observed in 1134 patients, but 2250 were provided treatment (see Table \ref{table:prias_summary}). Treatment in absence of GS7 may have been advised on the basis of PSA, number of biopsy cores with cancer, anxiety, or other reasons. We focused only on GS7 because of its strong association with cancer-related outcomes. Due to the periodical nature of biopsies, the time of GS7 was only known as a time interval in which it occurred.

\begin{table}
\small\sf\centering
\caption{\textbf{Patient characteristics for the PRIAS dataset}. The primary event of interest is Gleason~$\geq$~7. IQR: interquartile range, PSA: prostate-specific antigen.}
\label{table:prias_summary}
\begin{tabular}{lr}
\hline
\hline
Characteristic & Value\\
\hline
Total patients & 7813\\
Gleason~$\geq$~7 (primary event) & 1134\\
Treatment & 2250\\
Watchful waiting & 334\\
Loss to follow-up & 250\\
Death (unrelated to prostate cancer) & 95\\
Death (related to prostate cancer) & 2\\
\hline
Median age at diagnosis (years) & 61 (IQR: 66--71)\\
Median follow-up period per patient (years) &  1.8 (IQR: 0.9--3.99)\\
Total PSA measurements & 65798\\
Median number of PSA measurements per patient &  6 (IQR: 4--12)\\
Median PSA value (ng/mL) & 5.7 (IQR: 4.1--7.7)\\
Total biopsies & 15563\\
Median number of biopsies per patient &  2 (IQR: 1--3)\\
\hline
\end{tabular}
\end{table}