% !TEX root =  ../main_manuscript.tex 
\section{Discussion}
We successfully developed and externally validated a model for predicting upgrading-risk~\citep{bruinsma2017expert}, and providing risk-based personalized biopsy decisions, in prostate cancer AS.
Our work has four novel features over earlier risk calculators~\citep{coley2017prediction,ankerst2015precision}. First, our model was fitted to the world's largest AS dataset PRIAS and externally validated in the largest five cohorts of the Movember Foundation's GAP3 database~\citep{gap3_2018}. Second, the model predicts a patient's current and future upgrading-risk in a dynamic and personalized manner. Third, we use the risks to make a personalized schedule, and also calculate expected time delay in detecting upgrading (less is beneficial) if that schedule is followed. Thus, patients/doctors can compare schedules before making a choice. Fourth, we implemented our methodology in a user-friendly web-application (\url{https://emcbiostatistics.shinyapps.io/prias_biopsy_recommender/}) for PRIAS and validated cohorts.

Our PRIAS based model is useful for a large number of patients from the PRIAS and the following validation cohorts: Johns Hopkins AS (Hopkins), Memorial Sloan Kettering Cancer Center AS, King's College London AS (KCL), and Michigan Urological Surgery Improvement Collaborative AS (MUSIC). The model had a moderate AUC (0.55--0.75), a measure of discrimination, in all validation cohorts. In contrast, the mean absolute risk prediction error varied much more between cohorts. It was moderate in cohorts where the effect size for impact of PSA value and velocity on upgrading-risk was similar to that for PRIAS (e.g., Hopkins cohort). Otherwise, as in the case of KCL or MUSIC cohorts, the prediction error was large. Also, in cohorts with longer follow-up periods, prediction error improved over time as more follow-up data became available. Both KCL and MUSIC cohorts currently have a small follow-up period. Hence, we expect that prediction error will improve in the future with more data. Last, we required recalibration of our model's baseline hazard of upgrading, individually for all validation cohorts.

The clinical implications of our work are as follows. First, the cause-specific cumulative upgrading-risk at year five of follow-up was at most 50\% in all cohorts (Panel~B, Figure~\ref{fig:auc_beforecalib}). That is, many patients may not require any biopsy in the first five years of AS. Given the non-compliance and burden of frequent biopsies~\citep{bokhorst2015compliance}, the availability of our methodology as a web-application may encourage patients/doctors to consider upgrading-risk based personalized schedules instead. An additional advantage of these schedules is that they update as more patient data becomes available over follow-up. Furthermore, to assist patients/doctors in choosing between personalized and fixed schedules, the web-application provides a patient-specific estimate of time delay in detecting upgrading, for following both personalized and fixed schedules. We hope that this will objectively address patient apprehensions regarding adverse outcomes in AS. 

This work has certain limitations. Predictions for upgrading-risk and personalized schedules are available only for a currently limited, cohort-specific, follow-up period (Table 7, Supplementary~C). This problem can be mitigated by refitting the model with new follow-up data in the future. It is important to differentiate the instantaneous PSA velocity (predictor for upgrading-risk in our model), from the currently used constant PSA velocity. Unlike the drawbacks suffered by the constant PSA velocity~\citep{vickers2009psavelocity}, instantaneous PSA velocity is more precise. This is because it changes over time and is estimated from the fitted longitudinal PSA profile of a patient. Along with PSA, in some cohorts recently, MRI is also used for deciding biopsies. However, the utility of MRI can only be determined with more follow-up data in the future. Subsequently, MRI data can also be added as a predictor in our model. Decisions based on information combined from multiple sources can yield better results than based on MRI or PSA alone. We scheduled biopsies using cause-specific cumulative upgrading-risk. Accounting for competing events, such as treatment based on the number of positive biopsy cores, may lead to improved personalized biopsy decisions. Although, in this work, we did not consider such additional triggers for treatment because, unlike upgrading, they differ between cohorts~\citep{nieboer2018active}. Upgrading is susceptible to inter-observer variation too. Models which account for this variation~\citep{coley2017prediction,balasubramanian2003estimation} will be interesting to investigate further. However, the methodology for personalized scheduling, and for comparison of various schedules need not change.