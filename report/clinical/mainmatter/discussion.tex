% !TEX root =  ../main_manuscript.tex 
\section{Discussion}
We developed personalized schedules for repeat biopsies in PCa patients enrolled in AS programs. These schedules were based on a patient's risk for having a Gleason~$\geq$~7 (GS7). Patient- and visit-specific risks of GS7 were estimated using their entire history of PSA and repeat biopsies, and baseline characteristics. Consequently, the personalized schedules were updated as more data was gathered over follow-up. Risk calculators for GS7 are not new \citep{coley2017prediction,ankerst2015precision}. However, the novelty of our work is that we developed a methodology for scheduling personalized biopsies using those risks, as well a methodology to compare schedules, be it personalized or fixed, in simple terms of burden and benefit. More specifically, for each schedule we provided patients the times of biopsy and total biopsies (burden), and the delay in detection of GS7 (less is beneficial) expected due to that schedule. We also implemented our methodology in a web-application.

The proposed joint model accounted for the complex correlation structure that exists between longitudinal PSA measurements and time of GS7 of a patient. It also accounted for PSA measurements that were missing in patients who obtained GS7. This model adjusts the risks of GS7 upon a negative repeat biopsy. Thus complete patient information in consolidated into a single patient risk profile. Our model is fitted to the world's largest PCa AS program, PRIAS. We also validated our model predictions for GS7 in external cohorts that are part of the GAP3 database \citep{gap3_2018}. We found that the discrimination and calibration measures of the predictions from our model (Table~\ref{table:prediction_validation}) were similar for PRIAS (internal validation), and Toronto and Johns Hopkins cohorts (external validation). Given the large size of these cohorts, we expect that our model and the methodology will be useful to a large number of AS patients. Extending our model and methodology in other cohorts only requires fitting the model to their AS dataset.

The clinical implications of our work are as follows. The median survival time for GS7 is more than ten years in PRIAS, and in some other cohorts (Figure~\ref{fig:npmle_plot}). That is, more than 50\% of AS patients do not require any biopsy during the first ten years of follow-up. We hope that our work will address patient apprehensions regarding adverse outcomes in AS, in a more objective manner. Many AS programs still utilize a rigorous schedule of yearly biopsies \citep{nieboer2018active}. However, with concerns about non-compliance and burden of biopsies \citep{bokhorst2015compliance}, the availability of our web based tool may encourage patients and doctors to consider personalized schedules.

Our work has certain limitations. The proposed model is valid only for the first thirteen years of follow-up in PRIAS, whereas GS7 may occur much later in many patients. Due to this issue, the calibration and discrimination measures of predictions were also less accurate in later follow-up periods. These issue can be mitigated by refitting the model as more follow-up data is gathered in PRIAS. While we focused only on GS7, it is susceptible to inter-observer variation. Models which account for this variation \citep{coley2017prediction,balasubramanian2003estimation} will be interesting to investigate further. However, the methodology to schedule biopsies, and to estimate the consequences of following a schedule need not change. There is also a potential for including diagnostic information from novel biomarkers, quality of life measures, and magnetic resonance imaging (MRI). Currently, this data is very sparsely available in the PRIAS dataset. However, in future, adding this information in our model is trivial. This is because modeling correlation for extra outcomes, mainly entails connecting sub-models for the outcomes to shared random effects (see Figure~\ref{fig:jm_blockdiag}). Our model can also be used to schedule MRI scans, since they are expensive in developing countries.