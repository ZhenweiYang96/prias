% !TEX root =  ../main_manuscript.tex 
\section{Discussion}
We developed personalized schedules of biopsies for active surveillance (AS) prostate cancer patients. These schedules utilize a patient's risk of reclassification (increase in biopsy Gleason grade \citep{epsteinGG2014} from grade~1 to 2 or higher). Reclassification risk calculators are not new \citep{coley2017prediction,ankerst2015precision}. However, our work has three novel features. First, we personalized the risk of reclassification and it to schedule biopsies in a personalized manner. Second, we developed a methodology to assist patients and doctors to compare and choose between personalized and fixed schedules. Specifically, for each schedule we provided the timing and total number of biopsies (burden), and the expected time delay in detecting reclassification (smaller is beneficial). Third, we implemented our methodology in a web-application: \url{https://emcbiostatistics.shinyapps.io/prias_biopsy_recommender/}.

Our risk prediction model was a joint model \citep{rizopoulos2012joint,tsiatis2004joint}, fitted to the world's largest AS dataset, PRIAS. This model consolidated all available patient data, that is, historical PSA measurements and biopsy results, and baseline characteristics, into a single reclassification risk profile. This personalized risk profile and corresponding personalized schedules were updated as more patient data became available over follow-up.

We externally validated our model predictions for reclassification in five largest AS cohorts that are part of the GAP3 database \citep{gap3_2018}. We found that the AUC for predictions of GS7 over the follow-up period (Figure~\ref{fig:auc_pe}) was similar in external cohorts and PRIAS (internal validation). The RMSPE however was similar to PRIAS only for Memorial Sloan Kettering Cancer Center and Johns Hopkins cohorts. Given the large size of the latter two cohorts, we expect that our model and the methodology will be useful to a large number of AS patients. Extending our model and methodology in other cohorts only requires fitting the model to their AS dataset.

The clinical implications of our work are as follows. The median survival time for reclassification is more than ten years in PRIAS, and in some other cohorts (Figure~\ref{fig:npmle_plot}). That is, more than 50\% of AS patients do not require any biopsy during the first ten years of follow-up. We hope that our work will address patient apprehensions regarding adverse outcomes in AS, in a more objective manner. Many AS programs still utilize a rigorous schedule of yearly biopsies \citep{nieboer2018active}. However, with concerns about non-compliance and burden of biopsies \citep{bokhorst2015compliance}, the availability of our web based tool may encourage patients and doctors to consider personalized schedules.

Our work has certain limitations. The proposed model is valid only for the first ten years of follow-up in PRIAS, whereas reclassification may occur much later in many patients. In addition, model predictions were less accurate in later follow-up period due to lack of training data. These problems can be mitigated by refitting the model with new follow-up data in future. Although, we focused only on reclassification, it is susceptible to inter-observer variation. Models which account for this variation \citep{coley2017prediction,balasubramanian2003estimation} will be interesting to investigate further. However, the methodology for personalized scheduling, and for comparison of various schedules need not change. There is also a potential for including diagnostic information from novel biomarkers, quality of life measures, and magnetic resonance imaging (MRI). Currently, this data is very sparsely available in the PRIAS dataset. However, in future, our model can be extended by adding the new biomarkers as predictors.