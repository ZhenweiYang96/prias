\begin{abstract}
%225 words are maximum. Currently 225 words
Gold standard surveillance \textit{tests} for diagnosing disease \textit{progression} (biopsies, endoscopies, etc.) in early-stage chronic non-communicable disease patients (e.g.,~cancer, lung diseases) are usually invasive. For detecting progression timely, over their lifetime, patients undergo numerous invasive tests planned in a fixed one-size-fits-all manner (e.g.,~biannually). We present progression-risk based personalized test schedules that aim to balance better the number of tests (burden) and time delay in detecting progression (shorter is beneficial) than fixed schedules. \textcolor{Green}{Our motivation comes from the world's largest prostate cancer surveillance study PRIAS}.

Using joint models for time-to-event and longitudinal data, we consolidate auxiliary longitudinal data (e.g.,~biomarkers) and results of previous tests, into individualized future cumulative-risk of progression. We then create personalized schedules by planning tests on future visits where the predicted progression-risk is above a particular \textit{threshold} (e.g., 5\% risk). This schedule is updated on each follow-up with newly gathered data. To find the optimal risk threshold, we minimize a utility function of the expected number of tests (burden) and time delay in detecting progression (shorter is beneficial) for different thresholds. We estimate these two quantities in a patient-specific manner for following any schedule, by utilizing the predicted risk profile of the patient. Patients/doctors can employ these quantities to objectively compare various personalized and fixed schedules. \textcolor{Green}{Last, we implement our methodology in a web-application for real PRIAS study cancer patients}.
\end{abstract}