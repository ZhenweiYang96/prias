The burden of unnecessary biopsies is not only medical but also financial. \citet{keegan2012active} have shown that annual schedules can cost more than the treatment (brachytherapy or prostatectomy) to AS programs, and if biopsies were to be conducted every other year, then up to 28\% increase in savings per head from AS over treatment could be achieved.

At the time of inclusion in PRIAS, patients must have a Gleason score of six or less, DRE score of cT2c or less and a PSA of 10 ng/mL or less. When the Gleason score becomes greater than six, it is also known as Gleason reclassification (referred to as GR hereafter).

The time window $\Delta t$ can be automatically chosen as $\argmax_{\Delta t}\mbox{AUC}(t, \Delta t, s)$, the latter being a measure of discriminative capability of the model \citep{rizopoulosJMbayes,landmarking2017}. However such a time window may not be clinically relevant at all. 

\subsection{Results with $\kappa$ chosen on the basis of Youden's J}
In the main manuscript, for the personalized schedules based on dynamic risk of GR we chose $\kappa$ on the basis of $\mbox{F}_1$ score. However while conducting the simulation study, we also tried choosing it on the basis of Youden's $\mbox{J}$. Unlike $\mbox{F}_1$ score, Youden's $\mbox{J}$, is a measure of classification accuracy for both cases and controls. It is defined as:
\begin{align*}
\mbox{J}(t, \Delta t, s) &= \text{TPR}(t, \Delta t, s) - \text{FPR}(t, \Delta t, s), \quad \mbox{J}\epsilon [-1,1],\\
\text{TPR}(t, \Delta t, s) &= \mbox{Pr}\big\{\pi_j(t + \Delta t \mid t,s) \leq \kappa \mid T^*_j \epsilon (t, t + \Delta t]\big\},\\
\text{FPR}(t, \Delta t, s) &= \mbox{Pr}\big\{\pi_j(t + \Delta t \mid t,s) > \kappa \mid T^*_j > t + \Delta t \big\}.
\end{align*}
where $\mbox{TPR}(\cdot)$ and $\mbox{FPR}(\cdot)$ denote time dependent true positive rate (sensitivity) and false positive rate ($1 - \mbox{Specificity}$). The estimation for both is similar to the estimation of $\mbox{AUC}(t, \Delta t, s)$ given by \citet{landmarking2017}. The optimal value of $\kappa$ is $\argmax_{\kappa} \mbox{J}(t, \Delta t, s)$.

The simulation study results for Youden's $\mbox{J}$ were not presented in the main manuscript for brevity and are presented here in \ref{table : sim_study_pooled_estimates_extended}. In addition results for a hybrid approach between median time of GR and dynamic risk of GR based on Youden's $\mbox{J}$ are also presented. 

\subsubsection{Scheduling multiple biopsies}
\label{subsubsec : pers_sched_algorithm}
Scheduling biopsies using personalized schedules is an iterative process. Only one biopsy is scheduled at once, using all the information available up to that point in time. The information is manifested via the predictive distribution $g(T^*_j)$. It is important to note that new biopsy times are only proposed when the patient visits the hospital for a PSA measurement or for biopsy, since $g(T^*_j)$ is updated only at these time points. If a biopsy is conducted at time $t$, then scheduling the next biopsy at time $u \geq \text{max}(t,s)$ for patient $j$ is the goal. Although the predictive distribution $g(T^*_j)$ is updated with the new information $T^*_j > t$, there are a couple of restrictions on time $u$, namely:

\begin{enumerate}
\item Two consecutive biopsies should have a gap of at least 1 year between them. i.e. $u \geq t + 1$. Given the medical side effects of biopsies, the 1 year gap is strongly advised for patients enrolled in PRIAS. However, the personalized schedules do not take this rule into account.
\item Although it is required that $u \geq \text{max}(t,s)$, the personalized schedules may propose to perform a biopsy at a time $u \epsilon (t, s]$. The reason is that, GR is not a terminating event, and so the patients continue to visit for PSA measurements. The support of the predictive distribution $g(T^*_j)$, however does not depend on last time of PSA measurement $s$, and remains $(t, \infty)$.
\end{enumerate}
 
Lastly, it is extremely likely that on consecutive visits to the hospital for PSA measurements, the personalized biopsy time is postponed or preponed. The choice of time of biopsy from consecutive visits is not clear. To resolve these issues, we propose to supplement the personalized schedules with the following algorithm.


Let $\boldsymbol{\theta} = {\{\boldsymbol{\beta}^T, \boldsymbol{\gamma}^T, \boldsymbol{\alpha}^T, \sigma^2, \{\boldsymbol{D}[j,k] \mid j=k=1, \ldots q\}\}}^T$ be the vector of the parameters of the joint model.


For the relative risk model's parameters $\bmath{\gamma}$ and the association parameters $\bmath{\alpha}$, we use a global-local ridge-type shrinkage prior. For example, for the $s$-{th} element of $\bmath{\alpha}$ we assume (similarly for $\bmath{\gamma}$):
\begin{equation*} 
\alpha_s \sim \mathcal{N}(0, \tau\psi_s), \quad \tau^{-1} \sim \mbox{Gamma}(0.1, 0.1),  \quad \psi_s^{-1} \sim \mbox{Gamma}(1, 0.01).
\end{equation*} 
The global smoothing parameter $\tau$ has sufficiently mass near zero to ensure shrinkage, while the local smoothing parameter $\psi_s$ allows individual coefficients to attain large values. For the penalized version of the B-spline approximation to the baseline hazard, we use the following prior for parameters $\gamma_{h_0}$ \citep{lang2004bayesian}:
\begin{equation*}
p(\gamma_{h_0} \mid \tau_h) \propto \tau_h^{\rho(\bmath{K})/2} \exp\bigg(-\frac{\tau_h}{2}\gamma_{h_0}^T \bmath{K} \gamma_{h_0}\bigg),
\end{equation*}
where $\tau_h$ is the smoothing parameter that takes a Gamma(1, 0.005) hyper-prior in order to ensure a proper posterior for $\gamma_{h_0}$, $\bmath{K} = \Delta_r^T \Delta_r + 10^{-6} \bmath{I}$, where $\Delta_r$ denotes the $r$-th difference penalty matrix, and $\rho(\bmath{K})$ denotes the rank of $\bmath{K}$.