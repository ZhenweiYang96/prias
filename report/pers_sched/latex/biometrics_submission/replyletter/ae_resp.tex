% !TEX root =  ../ReplyLetterMain.tex
\clearpage
\section*{Response to Associate Editor's Comments}
We would like to thank the Associate Editor (AE) for his/her constructive comments, which have allowed us to considerably improve our paper. The main differences of the new version of the manuscript compared to the previous one can be found in Sections~5 and 6, Web Appendix A.2, C and D. In addition, changes regarding the specific comments have been made throughout the text.

You may find below our responses to the specific issues raised.

\begin{enumerate}
	\item \underline{How would delaying in activating treatments translate into survival outcomes?}

The AE raises a very important point. A long delay of the biopsy time may have repercussions with respect to the optimal timing of initiating active treatment. In the specific context of prostate cancer, and given the characteristics of the disease, delaying active treatment for 10-12 months is considered acceptable by the urologists. This is also what has driven the specific choices we have made for the optimal cut-off value for individualized risk predictions in Section \ref{subsec : estimation} of the original manuscript. However, in other contexts such a delay may not be acceptable, leading to another choice of the optimal cut-point. From a methodological perspective, accounting for the effect of delaying biopsy to survival would require modeling the survival outcome, and extending the relative risk sub-model in the specification of the joint model into a multi-state process. We are currently working on these extensions, but it falls outside the scope of our current paper.

\end{enumerate}

