% !TEX root =  ../main_manuscript.tex 
\section{Introduction}
\label{sec:introduction}
Prostate-specific antigen (PSA) based screening for prostate cancer has led to over-diagnosis of low and very low-grade prostate cancers~\citep{crawford2003epidemiology,newcomer1997temporal}. Many such cancers remain asymptomatic during a patient's lifetime. Hence, (over-) treatment upon diagnosis for low-grade cancers is nowadays substituted by prostate cancer active surveillance (AS). The goal of AS is to monitor patients continually and advise treatment only upon observing signs of cancer progression. To this end, tumors are evaluated periodically via PSA (ng/mL) blood test; digital rectal examination (DRE) for shape and size of the tumor; and biopsy Gleason grade group~\citep{epsteinGG2014}, a pathology report. Among these, the biopsy Gleason grade group is the strongest indicator of cancer-related outcomes. Consequently, a commonly used trigger for treatment in AS is Gleason upgrade, defined as increase in repeat biopsy Gleason grade group from group~1 to 2 or higher.

Most AS programs are about a decade old. Hence, the midterm pros and cons of AS have become apparent only recently. Among these, a crucial and contentious issue is the use of yearly biopsies for all patients~\citep{loeb2014heterogeneity}. Yearly biopsies help to detect Gleason upgrade timely (maximum delay of one year, Figure~\ref{fig:time_delay}). However, biopsies are painful, prone to complications, and recovery can take months~\citep{loeb2013systematic}. Yearly biopsies may be advantageous for patients who progress fast. Although, they also schedule many unnecessary biopsies for slow and/or non-progressing patients (50\% proportion in some AS programs). Moreover, patients do not always comply with such schedules~\citep{bokhorst2015compliance}. This may lead to delayed detection of Gleason upgrade and reduce the effectiveness of AS.

Our aim is to balance the number of biopsies (burden) and the time delay in the detection of Gleason upgrade (less is beneficial), better than fixed schedules. For this purpose, we intend to create personalized biopsy schedules that utilize patient-specific accumulated clinical data (age, historical PSA and DRE, and biopsy results). Previous alternatives to yearly biopsies can be divided into three categories. First, heuristic schedules such as biennial biopsies~\citep{inoue2018comparative}. Their drawback is that they ignore the difference in rate of Gleason upgrade between cohorts and/or patients (Web Figure...), and also over the follow-up period within the same cohort and/or patient. Second, personalized biopsy decisions employing partially observable Markov decision processes (POMDP); for prostate cancer~\citep{zhang2012optimization,barnett2018optimizing}, list for other cancers~\citep{alagoz2010operations}. Although, the efficacy of POMDP's is limited by the choice of reward function. Choosing rewards is difficult because even for the simplest of the POMDP's, infinite possible reward sets result into the same decision~\citep{vickers2006decision}. Third, personalized schedules can also be obtained by optimizing a loss function of clinical parameters of interest~\citep{bebu2017optimal,rizopoulos2015personalized}, including our previous work~\citep{tomer2019personalized}. However, our method scheduled only one biopsy at a time, and thus ignored the total burden of biopsies in a schedule. In addition, we utilized certain loss functions that inherently took clinically unacceptable high risks (e.g., 50\%) of Gleason upgrade. 

Our methodology in this work is as follows. First, we obtain a full specification of the joint distribution of the longitudinal PSA and DRE measurements, and time of Gleason upgrade (Section~\ref{sec:jointmodel}). We achieve this using joint models for time-to-event and longitudinal data~\citep{tsiatis2004joint,rizopoulos2012joint}. They are inherently personalized because they exploit patient-specific random effects to model all observed data. We fit our model to the data of the world's largest AS cohort, PRIAS, with 7813 patients from 100 medical centers worldwide, employing a common study protocol~\citep{bokhorst2016decade}. Subsequently, we use the fitted joint model to predict the personalized cumulative-risk of Gleason upgrade for new patients, over their whole follow-up period (Section~\ref{sec:schedule}). The predictions utilize their complete longitudinal PSA and DRE values, and previous biopsy results. We then propose biopsies on all those future follow-up visits where a patient's conditional cumulative-risk of Gleason upgrade is above a certain threshold (e.g., 10\% risk). We automate the choice of this threshold by optimizing a loss function of the two relevant features of a biopsy schedule, namely, the number of biopsies, and the expected time delay in detection of Gleason upgrade. The latter is also estimated in a patient-specific manner, for both fixed and personalized schedules. Thus patients and doctors can compare schedules before making a choice.

In Section~\ref{sec:demo} we briefly discuss the parameter estimates of the fitted joint model, and demonstrate the personalized schedules for real PRIAS patients. To compare the personalized and fixed schedules, we conduct an extensive and realistic simulation study based on a replica of the patients from the PRIAS cohort in Section~\ref{sec:sim_study}. We discuss the advantages and limitations of our work in Section~\ref{sec:discussion}.