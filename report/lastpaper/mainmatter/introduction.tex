% !TEX root =  ../main_manuscript.tex 
\section{Introduction}
\label{sec:introduction}
Chronic non-communicable diseases (e.g., cancer, renal, cardiovascular diseases, etc.) are the primary cause of human deaths worldwide~\citep{alwan2010monitoring}. Early identification of unfavorable changes in disease state (referenced as \textit{progression} hereafter) may prevent many of these deaths. To this end, surveillance tests are performed routinely for several diseases. Among all surveillance modalities, the most accurate or gold standard ones are often invasive. For example, biopsies, endoscopies, and colonoscopies are conducted repeatedly for diagnosing progression in prostate cancer~\citep{bokhorst2016decade}, Barrett's esophagus~\citep{streitz1993endoscopic}, and colorectal cancer~\citep{lieberman2012guidelines}, respectively. Similarly, repeat biopsies are employed to detect allograft deterioration in lung~\citep{mcwilliams2008surveillance} and kidney transplant~\citep{henderson2011surveillance} patients. 

Commonly, fixed schedules (e.g., every six months) are utilized for invasive tests. The frequency of tests varies between diseases and cohorts. Due to the periodical nature of schedules, progression is always detected with a time delay (see Figure). More frequent tests can lead to less time delay in detection of progression. However, invasive tests are difficult to conduct, are prone to severe complications, and cause patient discomfort. Furthermore, fixed schedules impose an equal medical burden, ignoring the speed of progression in patients. Hence, the frequency of invasive tests holds important implications for patients.

In this paper, we aim to balance the number of invasive tests (burden) and the time delay in detection of progression (less is beneficial) better than fixed schedules. For this purpose, we intend to create personalized test schedules that exploit patient-specific accumulated clinical data during follow-up. This data includes baseline characteristics of patients; results from previous invasive tests; and longitudinal biomarker, physical examination, and medical imaging measurements, etc. Previous approaches for personalized schedules may be divided into three categories. First, heuristic methods such as decision making flowcharts (citation here). Although, flowcharts discretize continuous clinical outcomes, often rely only the latest data measurement, and ignore the measurement error in observed outcomes. Second, personalized test decisions utilizing partially observable Markov decision processes~\citep{alagoz2010operations, steimle2017markov}. However, their application with continuous outcomes is limited by the curse of dimensionality. Third, personalized schedules obtained by optimizing a loss function of clinical parameters of interest~\citep{bebu2017optimal,rizopoulos2015personalized}, including our previous work on scheduling biopsies in prostate cancer~\citep{tomer2019personalized}. In this work, we will employ the third approach.

First, we develop a full specification of the joint distribution of patient-specific accumulated clinical data and time of \textit{progression}. We achieve this using joint models for time-to-event and longitudinal data~\citep{tsiatis2004joint,rizopoulos2012joint}. We exploit joint models because they are inherently personalized. Specifically, they exploit patient-specific random effects~\citep{laird1982random} to model longitudinal outcomes without discretizing them. We subsequently employ the fitted joint model for new patients, to estimate their patient-specific cumulative-risk profile over the whole follow-up period. These risk predictions utilize their complete observed clinical data. We then schedule invasive tests on all those future follow-up visits where a patient's conditional cumulative-risk of progression is above a certain threshold (e.g., 10\% risk). We automate the choice of this threshold and the resulting schedule. More specifically, we optimize a function of the number of tests in a schedule and the expected time delay in the detection of progression. We estimate this delay in a patient-specific manner for both fixed and personalized schedules, thus facilitating shared-decision of test schedules.

This research is motivated by the problem of scheduling biopsies~\citep{nieboer2018active} in the world's largest prostate cancer active surveillance (AS) study PRIAS~\citep{bokhorst2016decade}. It has 7813 patients from 100 medical centers worldwide, employing a common study protocol. These patients have low and very-low grade prostate cancer, often over-diagnosed due to prostate-specific antigen (PSA) based screening tests~\citep{crawford2003epidemiology}. The goal of AS is to delay serious treatments (e.g., surgery, chemotherapy, etc.) until cancer progression is observed. For this purpose, patients are monitored continually via PSA (ng/mL) blood tests, digital rectal examination (DRE) for shape and size of the tumor, and biopsy Gleason grade group~\citep{epsteinGG2014}. Since the latter is the strongest indicator of cancer-related outcomes, treatment is commonly advised upon observing an increase (cancer progression) in a patient's biopsy Gleason grade group. Currently, the most common biopsy schedule is yearly biopsies~\citep{loeb2014heterogeneity}. However, this leads to many unnecessary biopsies in slow/non-progressing patients (50\% proportion in some cohorts). Biopsy burden combined with patient non-compliance to frequent biopsies~\citep{bokhorst2015compliance} has raised concerns regarding the optimal biopsy schedule. Prostate cancer is the second most frequently diagnosed cancer in males worldwide~\citep{GlobalCancerStats2012}. Hence, biopsy schedules tailored for each AS patient can reduce the overall burden of biopsies in a large number of patients.

The rest of the paper is as follows. Section~\ref{sec:jointmodel} briefly introduces the joint modeling framework. In Section~\ref{sec:schedule} we present the methodology for personalized schedules, and then demonstrate them for biopsies in AS patients in Section~\ref{sec:results}. Lastly, in Section~\ref{sec:sim_study} we show the efficacy of personalized schedules for AS patients via a realistic simulation study.