% !TEX root =  ../main_manuscript.tex 
\section{Simulation Study}
\label{sec:sim_study}
Although we demonstrated personalized schedules for a real patient, we also intend to analyze and compare the efficacy of personalized and fixed schedules in a full study cohort. For this purpose, the ideal option of a real world randomized clinical trial is not possible. This is because, in all surveillance programs the true time of progression remains interval-censored. Second, multiple types of invasive tests cannot be scheduled for the same patient for both practical and ethical reasons. Hence, we conduct an extensive simulation study instead. To keep it realistic, we employ the prostate cancer active surveillance scenario. More specifically, we utilize the joint model fitted to the PRIAS cohort to generate simulated cohorts that are replicas of PRIAS. The simulated population has the same ten year follow-up period as the PRIAS study.

\subsection{Simulation Setup}
From this population, we first sample 500 datasets, each representing a hypothetical surveillance program with 1000 patients in it. We generate a true cancer progression time for each of the ${\mbox{500} \times \mbox{1000}}$ patients and then sample a set of DRE and PSA measurements at the same follow-up visit times as given in the PRIAS protocol. We then split each dataset into training (750 patients) and test (250 patients) parts, and generate a random and non‐informative censoring time for the training patients. We next fit a joint model of the specification given in Equations (\ref{eq:long_model_dre}), (\ref{eq:long_model_psa}), and (\ref{eq:rel_risk_model}) to each of the 500 training datasets and obtain MCMC samples from the 500 sets of the posterior distribution of the parameters.

In each of the 500 hypothetical AS programs, we utilize the corresponding fitted joint models to develop cancer progression risk profiles for each of the ${\mbox{500} \times \mbox{250}}$ test patients. We make the decision of biopsies for patients at their pre-scheduled follow-up visits for DRE and PSA measurements, on the basis of their estimated personalized cumulative risk of cancer progression. These decisions are made iteratively until a positive biopsy is observed. A recommended gap of one year between consecutive biopsies\cite{bokhorst2015compliance} is also maintained. Subsequently, for each patient, an entire personalized schedule of biopsies is obtained.

We evaluate and compare both personalized and currently practiced schedules of biopsies in this simulation study. Comparison of the schedules is based on the number of biopsies scheduled and the corresponding delay in the detection of cancer progression. We evaluate the following currently practiced fixed/heuristic schedules: biopsy annually, biopsy every one and a half years, biopsy every two years and biopsy every three years. We also evaluate the biopsy schedule of the PRIAS program. For the personalized biopsy schedules, we evaluate schedules based on three fixed risk thresholds: 5\%, 10\%, and 15\%, corresponding to a missed cancer progression being 19, 9, and 5.5 times more harmful than an unnecessary biopsy \cite{vickers2006decision}, respectively. We also implement a personalized schedule where for each patient, visit-specific risk thresholds are chosen using $\mbox{F}_1$ score.

\subsection{Results}