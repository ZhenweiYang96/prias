\begin{abstract}
%225 words are maximum. Currently 224 words
Commonly used gold standard surveillance \textit{tests} (biopsies, endoscopies, etc.) for confirming disease \textit{progression} in early-stage chronic non-communicable disease patients (cancer, cardiovascular, etc.) are invasive. For detecting progression timely (benefit), patients are exposed to numerous invasive tests repeatedly (burden) over their lifetime as per a fixed one-size-fits-all schedule. Motivated by this problem in the world's largest prostate cancer surveillance study PRIAS, we present disease progression risk-based personalized test schedules, that aim to better balance the number of invasive tests (burden), and time delay in detection of progression (less is beneficial) than fixed schedules.

%Patients diagnosed with early-stage chronic non-communicable diseases (cancer, cardiovascular, etc.) undergo invasive medical tests (biopsies, endoscopies, etc.), repeatedly for timely detection of disease \textit{progression}. Commonly, a fixed one-size-fits-all test schedule is employed for all patients. Consequently, many slow/non-progressing patients experience numerous unnecessary, burdensome tests over their lifetime. Motivated by this problem in the world's largest prostate cancer surveillance study PRIAS, we present personalized test schedules to balance the number of invasive tests (burden), and time delay in detection of progression (less is beneficial) better than fixed schedules. 
Using joint models for time-to-event and longitudinal data, we first consolidate auxiliary longitudinal data (e.g.,~biomarkers) and results of previous invasive tests, into individualized future cumulative-risk of \textit{progression}. Under this risk profile, we optimize a utility function of the number of tests and expected time delay in detection of progression to obtain a personalized schedule of invasive tests. Personalized schedules are updated as more patient
data becomes available over follow-up. We assist patients/doctors in comparing the consequences of opting for personalized versus fixed schedules objectively. For this, we exploit a patient's cumulative-risk profile to estimate the expected time delay in detection of progression for following both personalized and fixed schedules, in a patient-specific manner. We implement our methodology in a web-application for real prostate cancer patients of the PRIAS study.
\end{abstract}