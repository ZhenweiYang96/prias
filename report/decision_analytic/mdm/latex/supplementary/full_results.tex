% !TEX root =  ../supplementary.tex
\section{Full Results of the Simulation Study}

In the simulation study, we evaluate the following in-practice fixed/heuristic approaches \citep{loeb2014heterogeneity, inoue2018comparative} for biopsies: biopsy every year, biopsy every one and a half years, biopsy every two years and biopsy every three years. For the personalized biopsy approach we evaluate three fixed risk thresholds: 5\%, 10\% and 15\%, and a risk threshold chosen using $\mbox{F}_1$ score. Lastly, we also evaluate the PRIAS schedule of biopsies. We compare all the aforementioned schedules on two criteria, namely the number of biopsies they schedule and the corresponding delay in detection of cancer progression, in years (time of positive biopsy - true time of cancer progression). The corresponding results, using ${\mbox{500} \times \mbox{250}}$ test patients are presented in Table \ref{table:sim_study_all}.

\begin{table}[!htb]
\caption{\textbf{Simulation study results for all patients}: Estimated first, second (median), and third quartiles for number of biopsies ($\mbox{Q}^{\mbox{nb}}_1$, $\mbox{Q}^{\mbox{nb}}_2$, $\mbox{Q}^{\mbox{nb}}_3$) and for the delay in detection of cancer progression ($\mbox{Q}^{\mbox{delay}}_1$, $\mbox{Q}^{\mbox{delay}}_2$, $\mbox{Q}^{\mbox{delay}}_3$), in years, for various biopsy schedules. The delay is equal to the difference between the time of the positive biopsy and the unobserved true time of progression. The results in the table are obtained from  test patients of our simulation study.}
\label{table:sim_study_all}
\begin{tabular}{l|rrr|rrr}
\Hline
In-practice schedules & $\mbox{Q}^{\mbox{nb}}_1$ & $\mbox{Q}^{\mbox{nb}}_2$ & $\mbox{Q}^{\mbox{nb}}_3$ & $\mbox{Q}^{\mbox{delay}}_1$  & $\mbox{Q}^{\mbox{delay}}_2$  & $\mbox{Q}^{\mbox{delay}}_3$ \\
\hline
Every year (annual)         & 3  & 10 & 10 & 0.3 & 0.5 & 0.8 \\
Every 1.5 years      & 2  & 7  & 7  & 0.4 & 0.7 & 1.1 \\
Every 2 years       & 2  & 5  & 5  & 0.6 & 1.1 & 1.5 \\
Every 3 years      & 1  & 4  & 4  & 1.1 & 1.8 & 2.3 \\
PRIAS          & 2  & 4  & 6  & 0.3 & 0.7 & 1.0  \\
\hline
\multicolumn{7}{l}{Personalized approach}\\
\hline
Risk threshold: 5\%     & 2  & 6  & 8  & 0.3 & 0.6 & 0.9 \\
Risk threshold: 10\%    & 2  & 4  & 5  & 0.3 & 0.7 & 1.0   \\
Risk threshold: 15\%    & 2  & 3  & 4  & 0.4 & 0.8 & 1.4 \\
Risk using $\mbox{F}_1$ score & 1  & 2  & 3  & 0.5 & 0.9 & 2.2 \\
\hline
\end{tabular}
\end{table}

Since patients have varying cancer progression speeds, the impact of each schedule also varies with it. In order to highlight these differences we divide results for three types of patients, as per their time of cancer progression. They are \textit{fast, intermediate,} and \textit{slow progressing} patients. Although such a division may be imperfect and can only be done retrospectively in a simulation setting, we do it only for the purpose of illustration. We assume that the \textit{slow progressing} patients, are the 50\% of the total population, having a cancer progression time after the ten year follow-up period of the study (see Figure~\ref{fig:npmle_plot}). We assume \textit{fast progressing} patients, are the patients with an initially misdiagnosed state of cancer \citep{cooperberg2011outcomes}, or high risk patients who choose AS instead of immediate treatment. These are roughly 30\% of the population, having a cancer progression time less than 3.5 years. We label the remaining 20\% patients as \textit{intermediate progressing} patients.  Table \ref{table:fast}, Table \ref{table:intermediate}, and Table \ref{table:slow} show the results for the \textit{fast, intermediate,} and \textit{slow progressing} patients, respectively.

\begin{table}[!htb]
\caption{\textbf{Simulation study results for \textit{fast progressing} patients (30\% of all patients)}: Estimated first, second (median), and third quartiles for number of biopsies ($\mbox{Q}^{\mbox{nb}}_1$, $\mbox{Q}^{\mbox{nb}}_2$, $\mbox{Q}^{\mbox{nb}}_3$) and for the delay in detection of cancer progression ($\mbox{Q}^{\mbox{delay}}_1$, $\mbox{Q}^{\mbox{delay}}_2$, $\mbox{Q}^{\mbox{delay}}_3$), in years, for various biopsy schedules. The delay is equal to the difference between the time of the positive biopsy and the unobserved true time of progression. The results in the table are obtained from the \textit{fast progressing} test patients of our simulation study.}
\label{table:fast}
\begin{tabular}{l|rrr|rrr}
\Hline
In-practice schedules     & $\mbox{Q}^{\mbox{nb}}_1$ & $\mbox{Q}^{\mbox{nb}}_2$ & $\mbox{Q}^{\mbox{nb}}_3$ & $\mbox{Q}^{\mbox{delay}}_1$  & $\mbox{Q}^{\mbox{delay}}_2$  & $\mbox{Q}^{\mbox{delay}}_3$ \\
\hline
Every year (annual)         & 1  & 2  & 2  & 0.3 & 0.6 & 0.9 \\
Every 1.5 years      & 1  & 1  & 2  & 0.4 & 0.8 & 1.2 \\
Every 2 years       & 1  & 1  & 1  & 0.7 & 1.1 & 1.5 \\
Every 3 years      & 1  & 1  & 1  & 1.5 & 2.0   & 2.5 \\
PRIAS          & 1  & 2  & 2  & 0.3 & 0.7 & 1.0   \\
\hline
\multicolumn{7}{l}{Personalized approach}\\
\hline
Risk threshold: 5\%    & 1  & 2  & 2  & 0.3 & 0.6 & 0.9 \\
Risk threshold: 10\%    & 1  & 1  & 2  & 0.3 & 0.7 & 1.0   \\
Risk threshold: 15\%     & 1  & 1  & 2  & 0.4 & 0.8 & 1.2 \\
Risk using $\mbox{F}_1$ score & 1  & 1  & 2  & 0.5 & 0.9 & 2.0   \\
\hline
\end{tabular}
\end{table}

\begin{table}[!htb]
\caption{\textbf{Simulation study results for \textit{intermediate progressing} patients (20\% of all patients)}: Estimated first, second (median), and third quartiles for number of biopsies ($\mbox{Q}^{\mbox{nb}}_1$, $\mbox{Q}^{\mbox{nb}}_2$, $\mbox{Q}^{\mbox{nb}}_3$) and for the delay in detection of cancer progression ($\mbox{Q}^{\mbox{delay}}_1$, $\mbox{Q}^{\mbox{delay}}_2$, $\mbox{Q}^{\mbox{delay}}_3$), in years, for various biopsy schedules. The delay is equal to the difference between the time of the positive biopsy and the unobserved true time of progression. The results in the table are obtained from the \textit{intermediate progressing} test patients of our simulation study.}
\label{table:intermediate}
\begin{tabular}{l|rrr|rrr}
\Hline
In-practice schedules     & $\mbox{Q}^{\mbox{nb}}_1$ & $\mbox{Q}^{\mbox{nb}}_2$ & $\mbox{Q}^{\mbox{nb}}_3$ & $\mbox{Q}^{\mbox{delay}}_1$  & $\mbox{Q}^{\mbox{delay}}_2$  & $\mbox{Q}^{\mbox{delay}}_3$ \\
\hline
Every year (annual)         & 5  & 7  & 8  & 0.2 & 0.5 & 0.7 \\
Every 1.5 years      & 4  & 5  & 6  & 0.3 & 0.7 & 1.0   \\
Every 2 years       & 3  & 4  & 4  & 0.4 & 1.0   & 1.5 \\
Every 3 years      & 2  & 3  & 3  & 0.6 & 1.3 & 2.0  \\
PRIAS          & 3  & 5  & 6  & 0.3 & 0.7 & 1.3 \\
\hline
\multicolumn{7}{l}{Personalized approach}\\
\hline
Risk threshold: 5\%     & 5  & 6  & 7  & 0.3 & 0.6 & 0.9 \\
Risk threshold: 10\%    & 3  & 4  & 6  & 0.4 & 0.7 & 1.3 \\
Risk threshold: 15\%     & 3  & 3  & 5  & 0.4 & 0.8 & 1.7 \\
Risk using $\mbox{F}_1$ score & 2  & 3  & 5  & 0.5 & 1.0   & 2.4 \\
\hline
\end{tabular}
\end{table}

\begin{table}[!htb]
\caption{\textbf{Simulation study results for \textit{slow progressing} patients (50\% of all patients)}: Estimated first, second (median), and third quartiles for number of biopsies ($\mbox{Q}^{\mbox{nb}}_1$, $\mbox{Q}^{\mbox{nb}}_2$, $\mbox{Q}^{\mbox{nb}}_3$) and for the delay in detection of cancer progression ($\mbox{Q}^{\mbox{delay}}_1$, $\mbox{Q}^{\mbox{delay}}_2$, $\mbox{Q}^{\mbox{delay}}_3$), in years, for various biopsy schedules. The delay is equal to the difference between the time of the positive biopsy and the unobserved true time of progression. The results in the table are obtained from the \textit{slow progressing} test patients of our simulation study. Since no cancer progression is observed in the ten year follow-up period for these patients, delay cannot be estimated, and hence is not reported.}
\label{table:slow}
\begin{tabular}{l|rrr|rrr}
\Hline
In-practice schedules     & $\mbox{Q}^{\mbox{nb}}_1$ & $\mbox{Q}^{\mbox{nb}}_2$ & $\mbox{Q}^{\mbox{nb}}_3$ & $\mbox{Q}^{\mbox{delay}}_1$  & $\mbox{Q}^{\mbox{delay}}_2$  & $\mbox{Q}^{\mbox{delay}}_3$ \\
\hline
Every year (annual)         & 10 & 10 & 10 &   &   &   \\
Every 1.5 years      & 7  & 7  & 7  &   &   &   \\
Every 2 years       & 5  & 5  & 5  &   &   &   \\
Every 3 years      & 4  & 4  & 4  &   &   &  \\
PRIAS          & 4  & 6  & 8  &   &   &   \\
\hline
\multicolumn{7}{l}{Personalized approach}\\
\hline
Risk threshold: 5\%    & 6  & 7  & 9  &   &   &   \\
Risk threshold: 10\%    & 4  & 4  & 6  &   &   &   \\
Risk threshold: 15\%    & 2  & 3  & 4  &   &   &   \\
Risk using $\mbox{F}_1$ score & 2  & 2  & 4  &   &   &   \\
\hline
\end{tabular}
\end{table}