% !TEX root =  ../main_manuscript.tex
\section{Discussion}
\label{sec:discussion}
We proposed a methodology which better balances the number of biopsies, and the delay in detection of cancer progression than the currently practiced biopsy schedules, for low-risk prostate cancer patients enrolled in active surveillance (AS) programs. The proposed methodology combines a patient's observed DRE and PSA measurements, and the time of the latest biopsy, into a personalized cancer progression risk function. If the cumulative risk of cancer progression at a follow-up visit is above a certain threshold, then a biopsy is scheduled. We conducted an extensive simulation study, based on a replica of the patients from the PRIAS program, to compare this personalized approach for biopsies with the currently practiced biopsy schedules. We found personalized schedules to be dominant over many of the current biopsy schedules (see \hyperref[sec:results]{Results}).

The main reason for the better performance of personalized schedules is that they account for the variation in cancer progression rate between patients, and also over time within the same patient. In contrast, the existing fixed/heuristic schedules ignore that roughly 50\% of the patients never progress in the first ten years of follow-up (\textit{slow progressing} patients) and do not require biopsies. The \textit{fast progressing} patients require early detection. However, existing methods of identifying these patients, such as the use of PSA doubling time in PRIAS, inappropriately assume that PSA evolves linearly over time. Thus, they may not correctly identify such patients. The personalized approach, however, models the PSA profiles non-linearly. Furthermore, it appends information from PSA with information from DRE and previous biopsy results and combines them into a single a cancer progression risk function. The risk function is a finer quantitative measure than individual data measurements observed for the patients. In comparison to decision making with flowcharts, the risk as a single measure of a patient's underlying state of cancer may facilitate shared decision making for biopsies.

Existing work on reducing the burden of biopsies in AS primarily advocates less frequent heuristic schedules of biopsies \citep{inoue2018comparative} (e.g., biopsies biennially instead of annually). To our knowledge, risk-based biopsy schedules have barely been explored yet in AS \cite{nieboer2018active,bruinsma2016active}. The part of our results pertaining to the fixed/heuristic schedules is comparable with corresponding results obtained in existing work \cite{inoue2018comparative}, even though the AS cohorts are not the same. Thus, we anticipate similar validity for the results pertaining to the personalized schedules.

A limitation of the personalized approach is that the choice of risk threshold is not straightforward, as different thresholds lead to different combinations of the number of biopsies and the delay in detection of cancer progression. An approach is to choose a risk threshold which leads to personalized schedule dominant (e.g., 10\% risk) over the currently practiced schedules, for a given delay. Since personalized biopsy schedules are less burdensome, they may lead to better compliance. A second limitation is that the results that we presented are valid only in a 10 year follow-up period, whereas prostate cancer is a slow progressing disease. Thus more detailed results, especially for \textit{slow progressing} patients cannot be estimated. However, very few AS cohorts have a longer follow-period than PRIAS \cite{bruinsma2016active}. \textcolor{blue}{In a screening setting often the ethno-racial background of the patient, as well as the history of cancer in first degree relatives are checked. Our model does not take into account either. The reason is that the history of cancer in relatives been found to be predictive of cancer progression only in African-American patients \cite{goh2013clinical,telang2017prostate}. This is also evident by the fact that PRIAS and many other surveillance programs do not utilize this information in their biopsy protocols \cite{bokhorst2016decade,nieboer2018active}. In addition, patients who have a higher risk of an aggressive form of cancer are usually not recommended active surveillance. Hence the proposed model is relevant only for low-risk prostate cancer patients eligible for active surveillance. An exception is the active surveillance patients who are old and/or have comorbid illnesses. Currently, such patients may be removed from active surveillance and are instead offered the less intensive watchful waiting \cite{bokhorst2016decade} option. It is also possible to model watchful waiting as a competing risk in our model. However, this falls outside the scope of the current work because cancer progression as detected via biopsy is the standard trigger for treatment advice.} Lastly, our results are not valid when the patient data is missing not at random (MNAR).

There are multiple ways to extend the personalized decision making approach. For example, biopsy Gleason grading is susceptible to inter-observer variation \cite{coley2017}. Thus accounting for it in our model will be interesting to investigate further. To improve the decision making methodology, future consequences of a biopsy can be accounted for in the model by combining Markov decision processes with joint models for time-to-event and longitudinal data. There is also a potential for including diagnostic information from magnetic resonance imaging (MRI), such as the volume of the prostate tumor as a longitudinal measurement in our model. The resulting predictions can be used to the decide the time of next MRI as well as to make a decision of biopsy. The same holds true for the quality of life measures as well. However, given the scarceness of both MRI and quality of life measurements in the dataset, including them in the current model may not be feasible. We intend to further validate our results in a multi-center AS cohort, and subsequently develop a web application to assist in making shared decisions for biopsies.