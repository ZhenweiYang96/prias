% !TEX root =  ../main_manuscript.tex

\section{Discussion}
\label{sec:discussion}
Prostate cancer active surveillance (AS) programs schedule biopsies for patients to detect cancer progression. Biopsies are burdensome and hence each biopsy counts. However, currently there is no consensus on the best time interval for subsequent repeat biopsies \cite{loeb2014heterogeneity}. In order to reduce the delay in detection of cancer progression many AS programs schedule biopsies annually which leads to many unnecessary biopsies for slowly progressing patients. The world's largest AS program PRIAS attempts to identify these patients using their prostate-specific antigen (PSA) profile. However, despite their methodology compliance for biopsies is low. With an aim to better balance the burden of biopsies on patients and the delay in detection of cancer patients, in this article, we presented a methodology for personalizing the biopsy decision making process in AS programs. 

Our methodology utilizes joint models for time‐to‐event and longitudinal data. Existing approaches for biopsy schedules either discard information from PSA and digital rectal examination (DRE), or use crude measures such as PSA doubling time. In contrast, our proposed methodology makes a separate decision for each patient at each follow-up visit on the basis of finer measures such as patient specific instantaneous PSA value, PSA velocity, probability of having DRE larger than level T1c, and results from the previous biopsies. Our method combines the aforementioned measures into a patient and visit specific cancer progression risk function. It schedules a biopsy if the risk of cancer crosses a certain threshold. We compared our approach with the existing annual and PRIAS schedules, by conducting a realistic and extensive simulation study, for a 10 year follow-up period.

In the simulation study we found that the patients who never obtain cancer progression in the 10 year follow-up incur much burden due to currently used schedules. The PRIAS schedule, despite its effort to identify such patients using PSA doubling time, schedules a minimum of 4 biopsies and a median of 6 unnecessary biopsies for such patients. The annual schedule performs even worse by scheduling 10 unnecessary biopsies. In contrast, the personalized schedules that we proposed reduce it to a median of 3 to 4 biopsies depending upon the choice of the risk threshold. The choice of the risk threshold is also important. A low risk threshold such as 5\% risk may seem attractive to detect cancer progressions in time. However, it can work as worse as annual schedule in most situations, by scheduling unnecessary biopsies. A better idea is to either use a higher risk threshold or to automatically select it (see \hyperref[sec:methods]{Methods}). The personalized approach based on automatically chosen thresholds has the advantage of being generic for use in other active surveillance programs. In addition, it inherently accounts for the population specific cancer progression rate for the period in which patient has not been biopsied since last biopsy visit. 

For patients with faster and intermediate speed of cancer progression, we found in our simulation study that the personalized approach with automatically chosen threshold conducts less biopsies than the PRIAS schedule but still leads to a similar delay in detection of cancer progression. For slow, intermediate, and fast progressing patients combined, the aforementioned personalized approach schedules 3.75 biopsies before detecting a cancer progression. The personalized approach using 15\% risk schedules only 3 biopsies in total. These numbers are similar to the number of biopsies patients agree to undergo in PRIAS, if the non-compliance rates are also accounted in the PRIAS schedule. 

A disadvantage of our approach is that it does not create an entire optimal biopsy schedule based on a certain number of biopsies that a patient agrees to undergo beforehand. However given the fact that cancer progression information obtained from PSA and DRE measurements updates on each visit, and patients may also disagree with more biopsies over time, creating an optimal schedule may not always work.  
