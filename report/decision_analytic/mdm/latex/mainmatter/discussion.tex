% !TEX root =  ../main_manuscript.tex

\section{Discussion}
\label{sec:discussion}
In this paper we proposed a personalized methodology for making decisions for prostate cancer biopsies in active surveillance (AS) programs. Our methodology uses the entire history of prostate-specific antigen (PSA) levels and digital rectal examination (DRE) scores, and results of the latest biopsy, of each patient at each follow-up visit, to make a decision of biopsy at that visit. To this end, it utilizes joint models for time-to-event and longitudinal data. Using joint models, our method combines the observed data of a patient, into a personalized cancer progression risk profile. It schedules a biopsy if the cumulative risk of cancer progression at a follow-up visit is above a certain threshold.

The proposed personalized method has the following advantages over the in-practice fixed/heuristic schedules \citep{loeb2014heterogeneity,inoue2018comparative} of biopsies. Firstly, it accounts for varying cancer progression speeds of each patient. In contrast fixed/heuristic schedules, use a common schedule for all patients, which leads to many unnecessary biopsies in the case of \textit{slow progressing} (almost 50\% of all patients, see \hyperref[sec:results]{Results}) patients. Secondly, by providing urologists and patients an estimate of the risk of cancer progression, a more informed decision of biopsy can be made. The personalized method also has advantages over the in-practice PRIAS schedule of the world's largest AS program PRIAS. In PRIAS, the PSA profiles of patients are modeled linearly over time, which is contrary to observed PSA profiles of the patients (see Figure~6 in Appendix B). Consequently, PRIAS' method of scheduling more biopsies for only the \textit{fast progressing} patients, may not always correctly identify \textit{fast progressing} patients. To this end, we assume a non-linear profile for PSA, and utilize the complete observed data (historical PSA and DRE, time of latest biopsy) to develop the risk function (see Equation~\ref{eq:dynamic_risk_prob}), which is finer quantitative measure.

We compared the personalized approach with the in-practice biopsy schedules, by conducting a realistic and extensive simulation study. We evaluated the biopsy schedules on the basis of the number of biopsies they schedule, and the corresponding delay in detection of cancer progression. From the simulation study we found that both the personalized and PRIAS schedules provide much better balance between the number of biopsies and delay in detection of progression, than the fixed/heuristic approaches (see Figure~\ref{fig:better_balance_results}). Since, we conducted a simulation study, we are able to retrospectively check the impact of different schedules on patients with different speeds of cancer progression. In this regard, we observed that the commonly used annual schedule puts the highest burden on the \textit{slow progressing} patients (see Figure~\ref{fig:sim_res_combined}), by scheduling ten biopsies for each such patient. The PRIAS schedule, despite its effort to identify such patients using PSA doubling time, schedules a minimum of four and a median of six unnecessary biopsies for them. In contrast, the personalized biopsy decision making approach reduces it to a median of two to four biopsies depending upon the choice of the risk threshold. 

In this regard, a personalized approach with a 10\% risk threshold has a similar delay in detection of cancer progression as PRIAS, and it schedules less biopsies (see Figure~\ref{fig:sim_res_combined}). While, it may seem attractive for clinical use, but in light of the non-compliance results \citep{bokhorst2015compliance}, prescribing the same threshold to all patients may not be suitable. Patients have varying tolerance for the number of biopsies, and apprehensions about the delay in detection of cancer progression. Both of these must be considered while making the choice of a risk threshold. Furthermore, the choice should not be made only on the basis of point estimates such as median number of biopsies and/or median delay. Instead, measures of variance (see Figure~\ref{fig:sim_res_combined}) of both quantities should also be considered.

A limitation of our results is that, even if they are based on the world's largest AS program, other AS programs may differ in the patient characteristics\cite{inoue2018comparative}. In this regard, a simulation study based on a multi-center cohort is required. Currently, the follow-up of period of our study is ten years. More detailed results, especially for \textit{slow progressing} patients, can be obtained using a cohort having a longer follow-up period. Since Biopsy Gleason grading is susceptible to inter-observer variation \cite{Gleason_interobs_var}, accounting for it in our model will also be interesting to investigate further. There is also potential for including diagnostic information from magnetic resonance imaging (MRI) in our model. Lastly, considering quality of life measures while discussing utility of each personalized approach may also lead to better decision making. However, given the scarceness of such information in the dataset, including it in the current model may not be feasible.