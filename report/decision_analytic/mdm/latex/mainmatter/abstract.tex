% !TEX root =  ../main_manuscript.tex 

\begin{abstract}
\textbf{Background.} Low-risk prostate cancer patients enrolled in active surveillance programs commonly undergo biopsies for examination of cancer progression. Biopsies are conducted as per a fixed and frequent schedule (e.g., annual biopsies). Since biopsies are burdensome, patients do not always comply with the schedule, which increases the risk of delayed detection of cancer progression.

\textbf{Objective.} Our aim is to better balance the number of biopsies (burden) and the delay in detection of cancer progression (less is beneficial), by personalizing the decision of conducting biopsies.

\textbf{Data Sources.} We use patient data of the world's largest active surveillance program (PRIAS). It enrolled 5270 patients, had 866 cancer progressions, and an average of nine prostate-specific antigen (PSA) and five digital rectal examination (DRE) measurements per patient.

\textbf{Methods.} Using joint models for time-to-event and longitudinal data, we model the historical DRE and PSA measurements, and biopsy results of a patient at each follow-up visit. This results in a visit and patient-specific cumulative risk of cancer progression. If this risk is above a certain threshold, we schedule a biopsy. We compare this personalized approach with the currently practiced biopsy schedules via an extensive and realistic simulation study, based on a replica of the patients from the PRIAS program.

\textbf{Results.} The personalized approach saved a median of six biopsies (median:~4,~IQR:~2--5), compared to the annual schedule (median:~10,~IQR:~3--10). However, the delay in detection of progression (years) is similar for the personalized (median:~0.7,~IQR:~0.3--1.0) and the annual schedule (median:~0.5,~IQR:~0.3--0.8).

\textbf{Conclusions.} We conclude that personalized schedules provide substantially better balance in the number of biopsies per detected progression for men with low-risk prostate cancer.
\end{abstract}