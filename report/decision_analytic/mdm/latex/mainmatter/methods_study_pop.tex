\subsection{Study Population}
\label{subsec:study_population}
To develop our methodology we use data of prostate cancer patients from the world's largest AS study called PRIAS \cite{bokhorst2016decade}. More than 100 medical centers from 17 countries worldwide contribute to the collection of data, utilizing a common study protocol and a web-based tool, both available at \url{www.prias-project.org}. We use the data collected between December 2006 (beginning of the study) and December 2016. It consists of 5270 patients. Cancer progression is observed in 866 patients. For all patients, PSA measurements (ng/mL) are scheduled every 3 months for the first 2 years and every 6 months thereafter. The DRE measurements (ordinal scale) are scheduled every 6 months. We use the DRE measurements after converting them to a binary scale, namely $\mbox{DRE} > \mbox{T1c}$ and $\mbox{DRE} = \mbox{T1c}$. A DRE score of T1c\cite{schroder1992tnm} indicates a clinically inapparent tumor which is not palpable or visible by imaging. Tumors with $\mbox{DRE} > \mbox{T1c}$ are large enough to be palpable. On average 5 DRE and 9 PSA measurements have been recorded per patient. In order to identify cancer progression, biopsies are scheduled as per the PRIAS protocol (see \hyperref[sec:introduction]{Introduction}).