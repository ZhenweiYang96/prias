\caption{Boxplot showing variation in the number of biopsies, and the delay in detection of cancer progression, in years (time of positive biopsy - true time of cancer progression) for various biopsy schedules. Biopsies are conducted until cancer progression is detected. \textbf{Panel~A:} results for simulated patients who had a faster speed of cancer progression, with progression times between 0 and 3.5 years. \textbf{Panel~B:} results for simulated patients who had an intermediate speed of cancer progression, with progression times between 3.5 and 10 years. \textbf{Panel~C:} results for simulated patients who did not have cancer progression in the ten years of follow-up. \textbf{Types of personalized schedules:} Risk:~10\% and Risk:~5\% approaches, schedule a biopsy if the cumulative risk of cancer progression at a visit is more than 10\% and 5\%, respectively. Risk:~F1 works similar as previous, except that a visit-specific risk threshold is chosen by maximizing $\mbox{F}_1$ score (see \hyperref[sec:methods]{Methods}). Annual corresponds to a schedule of yearly biopsies and PRIAS corresponds to biopsies as per PRIAS protocol (see \hyperref[sec:introduction]{Introduction}).}