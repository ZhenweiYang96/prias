% !TEX root =  ../main.tex 

\section{Joint models for time to event and longitudinal outcomes}
\label{sec : jm_framework}
The first step in creating a personalized schedule for biopsies is to come up with a model for Gleason scores, PSA levels and other patient specific characteristics. In PRIAS, PSA is measured at the time of induction, every 3 months for the first 2 years in the study and then every 6 months thereafter. As for GR, the first time it is observed patients are taken off AS. Since our interest lies in finding the time of GR, we model it as a time to event outcome. We model the PSA levels as a longitudinal outcome. A joint model for time to event and longitudinal outcomes is used to model the association between the two types of outcomes. We next present a short introduction to the joint modeling framework we will use in this work.

\subsection{Joint model specification}
\label{subsec : jm_specification}
Let $T_i^*$ denote the true event time for the $i^{th}$ patient enrolled in an AS program. Let the vector of times at which biopsies are conducted for this patient be denoted by $T_i^b = \{T^b_{i0}, T^b_{i1}, \ldots T^b_{i{N_i^b}}; T^b_{ij} < T^b_{ik}, \forall j<k\}$, where $N_i^b$ are the total number of biopsies conducted. The true time of event $T_i^*$ cannot be observed directly and it is only known that it falls in an interval $(l_i, r_i]$, where $l_i = T^b_{i{N_i^b - 1}}, r_i = T^b_{i{N_i^b}}$ if the event (GR in the current context) is observed, and $l_i = T^b_{i{N_i^b}}, r_i=\infty$ if patient drops out of AS. The latter is also known as right censoring. Further let $\boldsymbol{y}_i$ denote the $n_i \times 1$ longitudinal outcome vector of the $i^{th}$ patient. The population of interest consists of all the patients enrolled in AS. For a sample of $n$ patients from this population the observed data is denoted by $\mathcal{D}_n = \{l_i, r_i, \boldsymbol{y}_i; i = 1, \ldots n\}$.\\

To model the evolution of the longitudinal measurements over time, the joint model utilizes a generalized linear mixed effects model. The distribution of longitudinal outcome $\boldsymbol{y}_i$ conditional on the random effects $\boldsymbol{b}_i$ is assumed to be a member of the exponential family. The corresponding linear predictor is given by:

\begin{equation*}
k[E\{y_i(t) \mid \boldsymbol{b}_i\}] = m_i(t) = \boldsymbol{x}_i^T(t) \boldsymbol{\beta} + \boldsymbol{z}_i^T(t) \boldsymbol{b}_i
\end{equation*}
where, $k(\cdot)$ denotes a known one-to-one monotonic link function and $y_i(t)$ denotes the value of the longitudinal outcome for patient $i$ at time $t$. The row vector of design matrix for fixed effects is denoted by $\boldsymbol{x}_i(t)$ and for random effects is denoted by $\boldsymbol{z}_i(t)$. Correspondingly the fixed effects are denoted by $\boldsymbol{\beta}$ and random effects by $\boldsymbol{b}_i$. The random effects are assumed to be normally distributed with mean zero and $q \times q$ covariance matrix $\boldsymbol{D}$.\\

To model the effect of longitudinal outcome on hazard of event, joint models utilize a relative risk sub-model. The hazard of event for patient $i$ at any time point $t$, denoted by $h_i(t)$, depends on a function of subject specific linear predictor $m_i(t)$ and/or the random effects:

\begin{align*}
h_i(t \mid \mathcal{M}_i(t), \boldsymbol{w}_i) &= \lim_{\Delta t \to 0} Pr\{t \leq T^*_i < t + \Delta t \mid T^*_i \geq t, \mathcal{M}_i(t), \boldsymbol{w}_i\}/\Delta t\\
&=h_0(t) \exp\{\boldsymbol{\gamma}^T\boldsymbol{w}_i + f\{M_i(t), \boldsymbol{b}_i, \boldsymbol{\alpha}\}\}
\end{align*}
where $\mathcal{M}_i(t) = \{m_i(v), 0\leq v \leq t\}$ denotes the history of the underlying longitudinal process up to time $t$. $\boldsymbol{w}_i$ is a vector of baseline covariates and $\boldsymbol{\gamma}$ are the corresponding parameters. The function $f(\cdot)$ parametrized by vector $\boldsymbol{\alpha}$ specifies the functional form \citep{brown2009assessing,rizopoulos2012joint,taylor2013real} of longitudinal outcome that is used in the linear predictor of the relative risk model. Some functional forms relevant to the problem at hand, and their interpretation are the following: 

\begin{itemize}
\item Association between hazard of event at time $t$ and longitudinal response at the same time point:\\
$f\{M_i(t), \boldsymbol{b}_i, \boldsymbol{\alpha}\} = \alpha m_i(t)$
\item Association between hazard of event at time $t$ and longitudinal response, as well as slope of longitudinal response at the same time point:\\
$f\{M_i(t), \boldsymbol{b}_i, \boldsymbol{\alpha}\} = \alpha_1 m_i(t) + \alpha_2 m'_i(t), \text{with}\  m'_i(t) = \frac{d m_i(t)}{dt}$
\end{itemize}

Lastly, $h_0(t)$ is the baseline hazard at time $t$, and is modeled flexibly using P-splines. More specifically:
\begin{equation*}
\log{h_0(t)} = \gamma_{h_0,0} + \sum_{q=1}^Q \gamma_{h_0,q} B_q(t, \boldsymbol{v})
\end{equation*}
where $B_q(t, \boldsymbol{v})$ denotes the $q^{th}$ basis function of a B-spline with knots $\boldsymbol{v} = v_1, \ldots v_Q$ and vector of spline coefficients $\gamma_{h_0}$. To avoid choosing the number and position of knots in the spline, a relatively high number of knots (e.g., 15 to 20) are chosen and the corresponding B-spline regression coefficients $\gamma_{h_0}$ are penalized using a differences penalty \citep{eilers1996flexible}.

\subsection{Parameter estimation}
\label{subsec : jm_param_estimation_bayesian}
We estimate parameters of the joint model using Markov chain Monte Carlo (MCMC) methods under the Bayesian framework. Let $\boldsymbol{\theta}$ denote the vector of the parameters of the joint model. The joint model postulates that given the random effects, time to event and longitudinal responses taken over time are all mutually independent. Under this assumption the posterior distribution of the parameters is given by:

\begin{align*}
p(\boldsymbol{\theta}, \boldsymbol{b} \mid \mathcal{D}_n) & \propto \prod_{i=1}^n p(l_i, r_i, \boldsymbol{y}_i \mid \boldsymbol{b}_i, \boldsymbol{\theta}) p(\boldsymbol{b}_i \mid \boldsymbol{\theta}) p(\boldsymbol{\theta})\\
& \propto \prod_{i=1}^n  \prod_{l=1}^{n_i} p(y_{il} \mid \boldsymbol{b}_i, \boldsymbol{\theta}) p(l_i, r_i \mid \boldsymbol{b}_i, \boldsymbol{\theta})  p(\boldsymbol{b}_i \mid \boldsymbol{\theta}) p(\boldsymbol{\theta})
\end{align*}
where the likelihood contribution of longitudinal outcome is:
\begin{equation*}
p(y_{il}\mid \boldsymbol{b}_i,\boldsymbol{\theta})=
\exp\Bigg\{\frac{y_{il}\psi_{il}(\boldsymbol{b}_i)-c\{\psi_{il}(\boldsymbol{b}_i)\}}{a(\phi)} -d( y_{il},\phi)         \Bigg\},
\end{equation*}
where $\psi_{il}(\boldsymbol{b}_i)$ and $\phi$ denote the natural and dispersion parameters in the exponential family, respectively, and $c(\cdot)$, $d(\cdot)$ and $a(\cdot)$ are known functions specifying the member of the exponential family. The likelihood contribution of the time to event outcome is given by:
\begin{equation}
\label{eq : likelihood_contribution_survival}
p\{l_i,r_i\mid \boldsymbol{b}_i,\boldsymbol{\theta}\} = \exp\Big\{-\int_0^{l_i} h_i(s \mid \mathcal{M}_i(s), \boldsymbol{w}_i)\,ds\Big\} - \exp\Big\{-\int_0^{r_i}h_i(s \mid \mathcal{M}_i(s), \boldsymbol{w}_i)\,ds\Big\}
\end{equation}
The integral in Equation (\ref{eq : likelihood_contribution_survival}) does not have a closed-form solution, and therefore we use a 15-point Gauss–Kronrod quadrature rule to approximate it.\\

We use independent normal priors with zero mean and variance 100 for the fixed effects $\boldsymbol{\beta}$ and inverse Gamma priors for scale parameters. For the variance–covariance matrix $\boldsymbol{D}$ of the random effects we take inverse Wishart prior with an identity scale matrix and degrees of freedom equal to the number $q$ of the random effects. For the relative risk model's parameters $\boldsymbol{\gamma}$ and the association parameters $\boldsymbol{\alpha}$, we use independent normal priors with zero mean and variance 100. For the penalized version of the B-spline approximation to the baseline hazard, we use the following prior for parameters $\gamma_{h_0}$ \citep{lang2004bayesian}:
\begin{equation*}
p(\gamma_{h_0} \mid \tau_h) \propto \tau_h^{\rho(\boldsymbol{K})/2} \exp\Big(-\frac{\tau_h}{2}\gamma_{h_0}^T \boldsymbol{K} \gamma_{h_0}\Big)
\end{equation*}
where $\tau_h$ is the smoothing parameter that takes a Gamma(1, 0.005) hyper-prior in order to ensure a proper posterior for $\gamma_{h_0}$, $\boldsymbol{K} = \Delta_r^T \Delta_r + 10^{-6} \boldsymbol{I}$, where $\Delta_r$ denotes the $r^{th}$ difference penalty matrix, and $\rho(\boldsymbol{K})$ denotes the rank of $\boldsymbol{K}$.