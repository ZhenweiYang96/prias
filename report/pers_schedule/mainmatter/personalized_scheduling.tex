\section{Personalized scheduling approaches}
\label{sec : personalized_scheduling}
Once a joint model for Gleason reclassification and PSA levels is obtained, the next step is to use it to create personalized schedules for biopsies. In this section we present the various personalized biopsy scheduling approaches and their motivation. The personalized schedules that we propose are dynamic in nature and thus at any given time, only 1 future biopsy is scheduled. The age of the patient and entire PSA, repeat biopsy history up to that time point is considered while computing the time of next biopsy. To elucidate the scheduling methods, we use patient $j$, introduced in detail in section \ref{subsubsec: dyn_surv_prob}, as a case.

\subsection{Conditional expected time to Gleason reclassification}
Given that the patient $j$ hadn't had Gleason reclassification up to time $t$, an estimate of true Gleason reclassification time, with zero bias and least squared loss is the conditional expected Gleason reclassification time: $E(T^*_j | T^*_j > t, \mathcal{Y}_j(s), D_n; \theta)$. The aforementioned two properties are also the motivation for using conditional expected Gleason reclassification time as the time of next biopsy. 

\begin{equation}
\label{eq : expectedFailureTime}
E(T^*_j | T^*_j > t, \mathcal{Y}_j(s), D_n; \theta) = t + \int_t^\infty \pi_j(u|t) \,du
\end{equation}

where, $\pi_j(u|t)$ is the dynamic survival probability (Section \ref{subsubsec: dyn_surv_prob}) for patient $j$ at time $u$. While expected time to Gleason reclassification is unbiased, it is more useful when the variance of time to Gleason reclassification, given by $Var(T^*_j | T^*_j >t, \mathcal{Y}_j(s), \mathcal{D}_n; \theta)$, is less. The variance is given by:

\begin{equation}
\label{eq : varFailureTime}
Var(T^*_j | T^*_j >t, \mathcal{Y}_j(s), \mathcal{D}_n; \theta) = 2 \int_t^\infty {(u-t) \pi_j(u|t) \,du} - {\bigg(\int_t^\infty \pi_j(u|t) \,du\bigg)}^2
\end{equation}

\subsection{Dynamic risk of failure}
\label{subsec : dynamic_risk_definitions}
When time to Gleason reclassification has a large variance (Eq. \ref{eq : varFailureTime}) then the offset (Section \ref{sec : introduction}) based on using conditional expected time to Gleason reclassification can vary wildly from patient to patient. In such a scenario a doctor/patient may want to perform biopsies earlier than the expected time to Gleason reclassification. This can be achieved by not crossing the risk of Gleason reclassification beyond a certain threshold.\\ 

The personalized scheduling approaches based on dynamic risk of reclassification, schedule biopsies at time points $u > t$ such that the dynamic risk of failure $1 - \pi_j(u|t)$ is higher than a certain threshold $1-\kappa$ beyond $u$. Or in other words the dynamic survival probability $\pi_j(u|t)$ is below a threshold $\kappa$ beyond $u$. The choice of the threshold $\kappa \epsilon [0,1]$ can be on the basis of amount of risk a patient or doctor is willing to take. It is also possible to automate the choice of $\kappa$. One such way is to choose a $\kappa$ for which a binary classification accuracy measure is maximized. In case of joint models, time dependent binary classification is more relevant because a patient can be in control group at some time $t_a$ and it can be in the cases at some future time point $t_b > t_a$. Based on \cite{rizopoulosJMbayes} we consider a subject $j$ to be a case if $\pi_j(t + \Delta t|t) \leq \kappa$ and a control if $\pi_j(t + \Delta t|t) > \kappa$. The time window $\Delta t$ can be either chosen on a clinical basis (such as 1 year in PRIAS) or it can be chosen at a point where $AUC(t, \Delta t)$ \cite{rizopoulosJMbayes} is largest. i.e. we chose a $\Delta t$ for which the model has the most discriminative capability at time $t$. Various binary classification accuracy measures can be maximized to select the cutoff $\kappa$. The ones we use in this report are:

\begin{itemize}
\item Max Sensitivity: $\underset{\kappa \epsilon [0, 1]} {\text{arg max Senstivity}}$,\\
where sensitivity is $P(\pi_j(t + \Delta t|t) \leq \kappa | T^*_j \epsilon (t, t + \Delta t])$ . 
\item Youden J Statistic: Sensitivity + Specificity - 1,\\
where the specificity is defined as $P(\pi_j(t + \Delta t|t) > \kappa | T^*_j > t + \Delta t)$.
\item Accuracy (ACC): $\dfrac{TP(t) + TN(t)}{TP(t) + FP(t) + TN(t) + FN(t)}$, where TP(t), FP(t), TN(t) and FN(t) are the number of true positives, false positives, true negatives and false negatives at time point $t$.
\item F1 Score: $\dfrac{2TP(t)}{2TP(t) + FP(t) + FN(t)}$, and it is derived by taking harmonic mean of precision and sensitivity.
\end{itemize}

We also compare $\kappa$ chosen by these automatic selection methods against a fixed $\kappa = 0.85$. The choice of 0.85 was arbitrary, assuming a scenario where a patient/doctor does not want to exceed 15\% risk of progression.
