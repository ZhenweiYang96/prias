\subsubsection{Scheduling multiple biopsies}
\label{subsubsec : pers_sched_algorithm}
Scheduling biopsies using personalized schedules is an iterative process. Only one biopsy is scheduled at once, using all the information available up to that point in time. The information is manifested via the predictive distribution $g(T^*_j)$. It is important to note that new biopsy times are only proposed when the patient visits the hospital for a PSA measurement or for biopsy, since $g(T^*_j)$ is updated only at these time points. If a biopsy is conducted at time $t$, then scheduling the next biopsy at time $u \geq \text{max}(t,s)$ for patient $j$ is the goal. Although the predictive distribution $g(T^*_j)$ is updated with the new information $T^*_j > t$, there are a couple of restrictions on time $u$, namely:

\begin{enumerate}
\item Two consecutive biopsies should have a gap of at least 1 year between them. i.e. $u \geq t + 1$. Given the medical side effects of biopsies, the 1 year gap is strongly advised for patients enrolled in PRIAS. However, the personalized scheduling methods do not take this rule into account.
\item Although it is required that $u \geq \text{max}(t,s)$, the personalized scheduling methods may propose to perform a biopsy at a time $u \epsilon (t, s]$. The reason is that, GR is not a terminating event, and so the patients continue to visit for PSA measurements. The support of the predictive distribution $g(T^*_j)$, however does not depend on last time of PSA measurement $s$, and remains $(t, \infty)$.
\end{enumerate}
 
Lastly, it is extremely likely that on consecutive visits to the hospital for PSA measurements, the personalized biopsy time is postponed or preponed. The choice of time of biopsy from consecutive visits is not clear. To resolve these issues, we propose to supplement the personalized scheduling methods with the following algorithm.


Let $\boldsymbol{\theta} = {\{\boldsymbol{\beta}^T, \boldsymbol{\gamma}^T, \boldsymbol{\alpha}^T, \sigma^2, \{\boldsymbol{D}[j,k] \mid j=k=1, \ldots q\}\}}^T$ be the vector of the parameters of the joint model.