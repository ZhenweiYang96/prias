% !TEX root =  ../main.tex 

\section{Discussion}
\label{sec: discussion}

In this paper we presented personalized biopsy scheduling methods for patients enrolled in AS programs. The problem at hand was that AS patients have to undergo repeat biopsies frequently, which causes medical side effects and also brings financial burden. The existing schedules such as PRIAS schedule also had high patient non-compliance because of frequent biopsies and crude analysis of PSA. To approach these problems, we first came up with a joint model to combine the information from PSA as well as repeat biopsies in a more sophisticated manner than the existing PRIAS schedule. Secondly, using the information from the model, we proposed personalized schedules tailored for individual patients. In total we proposed 2 different class of personalized schedules: those based on central tendency of the distribution of time of GR for individual patient, and those based on dynamic risk of GR. In addition we also proposed a combination (mixed approach) of these 2 approaches. Lastly, we also proposed criteria for evaluation of various scheduling methods.\\

We demonstrated using PRIAS data set that the personalized schedules adjust the time of biopsy with results from repeat biopsies and PSA profile even when the two are not in complete concordance with each other. Secondly from the simulation study we observed that personalized scheduling method based on dynamic risk of GR ($\text{F}_1$-Score) performed better than PRIAS schedule in terms of both mean and variance of number of biopsies and offset. We also observed that schedule based on expected and median time to GR conducted only 2 biopsies on average, which is very promising compared to PRIAS and annual schedule which conducted 5 biopsies on average. In addition, for the former two methods approximately 92\% of the patients had an offset less than 36 months, which is the maximum acceptable offset in PRIAS. If a stronger restriction is prescribed for offset, then we propose that the mixed approach be used since it offers the best of the two worlds. i.e. not too many biopsies and not too high offset. While each of the personalized methods has their own disadvantages, given the different ones, they also provide an option for the AS programs to choose as per their requirements, in lieu of a fixed common schedule for all patients. In this regard, there is potential to develop and analyze personalized schedules using loss functions which asymmetrically penalize overshooting/undershooting the target GR time. Lastly, while in this work we considered that GR time was interval censored, the Gleason scores are susceptible to inter-observer variation. Models and schedules which account for error in measurement of time of GR, will be interesting to investigate further.