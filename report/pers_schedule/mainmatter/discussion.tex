% !TEX root =  ../main.tex 

\section{Discussion}
\label{sec: discussion}
In this paper we presented personalized biopsy scheduling methods for patients enrolled in AS programs. The problem at hand was that the AS patients have to undergo repeat biopsies frequently, which causes medical side effects and also brings financial burden. On top of that the existing schedules such as PRIAS schedule had high patient non-compliance because of frequent biopsies and crude analysis of PSA. To approach these problems, we first came up with a joint model to combine the information from PSA as well as repeat biopsies in a more sophisticated manner than the existing PRIAS schedule. Secondly, using the information from the model, we proposed personalized schedules tailored for individual patients. We proposed 2 different class of personalized schedules: those based on central tendency of the distribution of time of GR for individual patient, and those based on dynamic risk of GR. In addition we also proposed a combination (mixed approach) of these 2 approaches. Lastly, we also proposed criteria for evaluation of various scheduling methods.\\

We demonstrated using PRIAS data set that the personalized schedules adjust the time of biopsy with results from repeat biopsies and PSA profile even when the two are not in complete concordance with each other. Secondly from the simulation study we observed that personalized scheduling method based on dynamic risk of GR ($\text{F}_1$-Score) performed better than PRIAS schedule in terms of both mean and variance of number of biopsies and offset. We also observed that the PRIAS schedule performed quite similar to the annual schedule and the latter may be preferred over it since the latter only conducted 0.38 biopsies more on average but it always detected GR within an year of occurrence. The schedules based on expected and median time to GR conducted only 2 biopsies on average, which is very promising compared to PRIAS and annual schedule which conducted 4.9 and 5.2 biopsies on average respectively. In addition, for schedules based on expected and median time of GR, approximately 92\% of the patients had an offset less than 36 months, which is the maximum acceptable offset in PRIAS. If a stronger restriction is prescribed for the offset, then we propose that the mixed approach be used since it offers the best of the two worlds. i.e. not too many biopsies and not too high offset. Lastly, we observed that the personalized methods performed the same for all sub-groups in our population, however the performance of annual and PRIAS schedule was dependent on the failure times of the patients.\\

While each of the personalized methods has their own disadvantages and advantages, they also offer multiple choices to the AS programs to choose the one as per their requirements, instead of choosing a common fixed schedule for all patients. In this regard, there is potential to develop and analyze more personalized schedules. For e.g. using loss functions which asymmetrically penalize overshooting/undershooting the target GR time can be interesting. Another option is to choose $\kappa$ on the basis of other binary classification accuracy measures which were not discussed in this paper. Lastly, in this work we considered that GR time was interval censored, however the Gleason scores are susceptible to inter-observer variation. Models and schedules which account for error in measurement of time of GR, will be interesting to investigate further.