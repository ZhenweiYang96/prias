% !TEX root =  ../main.tex 

\subsection{Posterior predictive distribution for time to GR}
\label{subsec : ppd_time_to_GR}
Let $\mathcal{Y}_j(s)$ denote the history of PSA levels taken up to time $s$ for patient $j$. The information from PSA history and repeat biopsies is manifested by the posterior predictive distribution $g(T^*_j)$, given by (conditioning on baseline covariates $\boldsymbol{w}_i$ is dropped for notational simplicity hereafter):
\begin{equation}
\label{eq : dyn_dist_fail_time}
\begin{split}
g(T^*_j) &= p\big(T^*_j \mid T^*_j > t, \mathcal{Y}_j(s), \mathcal{D}_n\big)\\
&= \int p\big(T^*_j \mid T^*_j > t, \mathcal{Y}_j(s), \boldsymbol{\theta}\big) p\big(\boldsymbol{\theta} \mid \mathcal{D}_n\big) \diff \boldsymbol{\theta}\\
&= \int \int p\big(T^*_j \mid T^*_j > t, \boldsymbol{b_j}, \boldsymbol{\theta}\big) p\big(\boldsymbol{b}_j \mid T^*_j>t, \mathcal{Y}_j(s), \boldsymbol{\theta}\big)p\big(\boldsymbol{\theta} \mid \mathcal{D}_n\big) \diff \boldsymbol{b}_j \diff \boldsymbol{\theta}
\end{split}
\end{equation}
The posterior predictive distribution depends on the observed longitudinal history via the random effects $\boldsymbol{b_j}$. The posterior distribution of the parameters $\boldsymbol{\theta}$, denoted by $p(\boldsymbol{\theta} \mid \mathcal{D}_n)$ is obtained as in Section \ref{subsec : jm_param_estimation_bayesian}.