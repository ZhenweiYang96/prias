% !TEX root =  ../main.tex 

\section{Introduction}
\label{sec : introduction}
Prostate cancer is the development of cancer in the prostate gland. With increase in number of screening tests and increase in life expectancy, an increase in diagnosis of low grade prostate cancers has been observed. Majority of these cancers have good long-term survival. In many cases the prostate cancer is (over) diagnosed solely due of screening. i.e. it wouldn't have shown any malignant symptoms for a long time otherwise. Because of these reasons prostate cancer patients are often motivated to join active surveillance (AS) programmes instead of taking immediate treatment. The goal of AS programmes is to routinely check the progression of prostate cancer and avoid serious treatments such as surgery or chemotherapy as long as they are not needed. One such AS programme is ErasmusMC's PRIAS. Patients in PRIAS are closely monitored using serum PSA levels and repeat biopsies. Biopsies are evaluated using the Gleason grading system. Gleason scores range between 2 and 10, with 10 corresponding to a very serious state of prostate cancer. Patients who join PRIAS have a Gleason score of 6 or less, and a PSA of 10 ng/mL or less at the time of induction. While both high PSA and high Gleason score are indicators of prostate cancer progression, Gleason scores are considered to be the gold standard in this regard. If in any biopsy, a Gleason score greater than 6 is observed, then it is considered that the disease has progressed and the patient is removed from AS for further curative treatment. When Gleason score becomes greater than 6, it is also known as Gleason reclassification.\\

The reliability of Gleason scores comes at a high cost. Biopsies are difficult to obtain, are painful and have serious side effects for prostate cancer patients. So much so, that PRIAS strongly adheres to the rule of not having more than 1 biopsy per year. For patients enrolled in PRIAS, one biopsy is performed at the time of induction, and the rest are scheduled at 1, 4, 7, 10, 15, 20, ... years thereafter. For patients who have a PSA doubling time of less than 10 years, repeat biopsy every year is advised. Such a schedule however is not suitable for every patient. Patients whose cancers progress slowly, often end up having biopsies when they are not needed. For patients who have faster progressions, PSA doubling time is used as a crude measure to decide if frequent biopsies should be performed. \\

An ideal biopsy schedule is the one with the least number of biopsies and the smallest offset. We define offset as difference between the time at which biopsy results show that Gleason reclassification happened and the actual time at which Gleason reclassification took place. The search for an ideal biopsy schedule is also the motivation behind this work. In this report we present methods to aperiodically schedule biopsies for prostate cancer patients who are on active surveillance (AS). Our methods create a personalized schedule of biopsies for every patient. The personalized schedule not only considers baseline factors such as age of patient, but also considers evolution of PSA over time. The use of PSA in creating a personalized schedule is important because PSA is very easy to measure and the measurement procedure does not have any side effects. Further, the personalized schedule also utilizes the results from repeat biopsies and the information about progression of prostate cancer, as measured by Gleason scores, over time.\\

In this report we compare the personalized schedules against the schedule of PRIAS program, for the offset and number of biopsies conducted. We also compare personalized schedules against the most aggressive biopsy schedule of doing a biopsy every year. Using a simulation study we demonstrate that personalized schedules are a very promising method to decide biopsy times for prostate cancer patients.