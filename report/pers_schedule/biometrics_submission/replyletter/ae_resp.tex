% !TEX root =  ../ReplyLetterMain.tex
\clearpage
\section*{Response to AE's Comments}
We would like to thank the Associate Editor for his/her constructive comments, which have allowed us to considerably improve our paper. The main differences of the new version of the manuscript compared to the previous one can be found in Sections~2, 3 and 5. In addition, changes regarding the specific comments have been made throughout the text.

With respect to the AE's note about how delaying active treatment relates to survival outcome, the biopsies are already conducted for the patients according to the PRIAS schedule. Thus we cannot test the effect of delay in detection Gleason reclassification (GR) due to personalized schedules on the prostate cancer (PCa) specific mortality, unless they are applied on the patients. It may however be possible to compare the effect of delay in detection of GR due to PRIAS (less frequent biopsies) and annual schedules (more frequent biopsies), on the PCa specific mortality. However this would be out of the scope of our current focus on personalized schedules. In addition, multiple studies have reported small PCa specific mortality in low risk patients enrolled in active surveillance programs \citep{loeb2016immediate,tosoian2011active,klotz2009clinical}. That is, less frequent schedules may be useful in such scenarios. For example, for slowly progressing patients (subgroup $G_3$) in our simulation study, we observed that even a personalized schedule which conducts on average two biopsies leads to an average delay of 10 months in detecting GR. This is only four months more delay than that of the annual schedule. Given, the low PCa mortality a relative difference of four months may not be that bad an alternative.
