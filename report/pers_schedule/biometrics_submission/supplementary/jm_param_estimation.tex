%Examples here 
%http://www.biometrics.tibs.org/datasets/datasets.html#Sep67

\section{JM Parameter estimation}
\label{sec : jm_param_estimation_bayesian}
We estimate parameters of the joint model using Markov chain Monte Carlo (MCMC) methods under the Bayesian framework. Let $\bmath{\theta}$ denote the vector of the parameters of the joint model. The joint model postulates that given the random effects, time to GR and longitudinal responses taken over time are all mutually independent. Under this assumption the posterior distribution of the parameters is given by:
\begin{align*}
p(\bmath{\theta}, \bmath{b} \mid \mathcal{D}_n) & \propto \prod_{i=1}^n p(l_i, r_i, \bmath{y}_i \mid \bmath{b}_i, \bmath{\theta}) p(\bmath{b}_i \mid \bmath{\theta}) p(\bmath{\theta})\\
& \propto \prod_{i=1}^n p(l_i, r_i \mid \bmath{b}_i, \bmath{\theta}) p(\bmath{y_i} \mid \bmath{b}_i, \bmath{\theta}) p(\bmath{b}_i \mid \bmath{\theta}) p(\bmath{\theta})
\end{align*}
where the likelihood contribution of longitudinal outcome conditional on random effects is:
\begin{align*}
p(\bmath{y_i} \mid \bmath{b}_i, \bmath{\theta}) &= \frac{1}{\big(\sqrt{2 \pi \sigma^2}\big)^{n_i}} \exp\bigg\{-\frac{{\lVert{\bmath{y_i} - \bmath{X}_i\bmath{\beta} - \bmath{Z}_i\bmath{b}_i}\rVert}^2}{\sigma^2}\bigg\},\\
\bmath{X}_i &= \{\bmath{x}_i(t_{i1})^T, \ldots, \bmath{x}_i(t_{in_i})^T\}^T,\\
\bmath{Z}_i &= \{\bmath{z}_i(t_{i1})^T, \ldots, \bmath{z}_i(t_{in_i})^T\}^T
%p(\bmath{b}_i \mid \bmath{\theta}) = \frac{1}{\sqrt{(2 \pi)^q \text{det}(\bmath{D})}} \exp\{\bmath{b}_i^T \bmath{D}^{-1} \bmath{b}_i\} 
\end{align*}
The likelihood contribution of the time to GR outcome is given by:
\begin{equation}
\label{eq : likelihood_contribution_survival}
p\{l_i,r_i\mid \bmath{b}_i,\bmath{\theta}\} = \exp\Big\{-\int_0^{l_i} h_i(s \mid \mathcal{M}_i(s), \bmath{w}_i)\rmn{d}{s}\Big\} - \exp\Big\{-\int_0^{r_i}h_i(s \mid \mathcal{M}_i(s), \bmath{w}_i)\rmn{d}{s}\Big\}
\end{equation}
The integral in (\ref{eq : likelihood_contribution_survival}) does not have a closed-form solution, and therefore we use a 15-point Gauss–Kronrod quadrature rule to approximate it.

We use independent normal priors with zero mean and variance 100 for the fixed effects $\bmath{\beta}$ and inverse Gamma priors for $\sigma^2$. For the variance–covariance matrix $\bmath{D}$ of the random effects we take inverse Wishart prior with an identity scale matrix and degrees of freedom equal to the number $q$ of the random effects. For the relative risk model's parameters $\bmath{\gamma}$ and the association parameters $\bmath{\alpha}$, we use independent normal priors with zero mean and variance 100. For the penalized version of the B-spline approximation to the baseline hazard, we use the following prior for parameters $\gamma_{h_0}$ \citep{lang2004bayesian}:
\begin{equation*}
p(\gamma_{h_0} \mid \tau_h) \propto \tau_h^{\rho(\bmath{K})/2} \exp\bigg\{-\frac{\tau_h}{2}\gamma_{h_0}^T \bmath{K} \gamma_{h_0}\bigg\}
\end{equation*}
where $\tau_h$ is the smoothing parameter that takes a Gamma(1, 0.005) hyper-prior in order to ensure a proper posterior for $\gamma_{h_0}$, $\bmath{K} = \Delta_r^T \Delta_r + 10^{-6} \bmath{I}$, where $\Delta_r$ denotes the $r$-th difference penalty matrix, and $\rho(\bmath{K})$ denotes the rank of $\bmath{K}$.