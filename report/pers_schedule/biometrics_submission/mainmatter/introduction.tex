% !TEX root =  ../pers_schedules.tex 
\section{Introduction}
\label{sec : introduction}
In this decade prostate cancer is the second most frequently diagnosed cancer (14\% of all cancers) in males worldwide, with nearly 67\% of all prostate cancer cases reported in developed countries \citep{GlobalCancerStats2012}. The increase in diagnosis of low grade prostate cancers has been attributed to increase in life expectancy and increase in number of screening programs \citep{potoskyPSAcancer}. A major issue of screening programs that also has been established in other types of cancers (e.g., breast cancer) is over-diagnosis. To avoid overtreatment, patients diagnosed with low grade prostate cancer are often motivated to join active surveillance (AS) programs. The goal of AS is to routinely examine the progression of prostate cancer and avoid serious treatments such as surgery or chemotherapy as long as they are not needed.

Currently the largest AS program worldwide is Prostate Cancer Research International Active Surveillance, also known as PRIAS \citep{bokhorst2016decade}. Patients enrolled in PRIAS are closely monitored using serum prostate-specific antigen (PSA) levels, digital rectal examination (DRE) and repeat prostate biopsies. Results from biopsies are graded on a scale called Gleason, which takes values between 2 and 10, with 10 corresponding to a serious state of the disease. At the time of induction in PRIAS, patients must have a Gleason score of six or less, DRE score of cT2c or less and a PSA of 10 ng/mL or less. In PRIAS, a PSA doubling time (PSA-DT) of less than three years indicates prostate cancer progression, where PSA-DT is measured as the inverse of the slope of regression line through the base two logarithm of PSA values. However until either DRE or Gleason are observed to be higher than the aforementioned thresholds, patients are not removed from AS for curative treatment \citep{bokhorst2016decade}. When the Gleason score becomes greater than six, it is also known as Gleason reclassification (referred to as GR hereafter).

Biopsies are reliable but they are also difficult to conduct, cause pain and have serious side effects such as hematuria and sepsis \citep{loeb2013systematic}. Due to these reasons, majority of the AS programs worldwide strongly advise at most one biopsy per year. Conducting a biopsy annually (we refer to it as annual schedule hereafter) has the advantage that GR can be detected within one year of its occurrence. This may work well for patients with a faster progressing disease, however for patients with a slower progressing disease many unnecessary biopsies are scheduled. The drawbacks of unnecessary biopsies are not only medical but also financial. \citet{keegan2012active} have shown that annual schedules can cost more than the treatment (brachytherapy or prostatectomy) to AS programs, and if biopsies were to be conducted every other year, then up to 28\% increase in savings per head from AS over treatment could be achieved. Despite this, several AS programs employ the annual schedule \citep{tosoian2011active,welty2015extended}.

Scheduling frequent biopsies for patients has also lead to a high non-compliance rate for repeat biopsies \citep{bokhorst2015compliance}, which reduces the effectiveness of AS programs because progression is detected late. In PRIAS some of the reasons reported for non-compliance were: \textquoteleft patient does not want biopsy\textquoteright, \textquoteleft complications on last biopsy\textquoteright{} and \textquoteleft no signs of disease progression on previous biopsy\textquoteright. The PRIAS schedule and the compliance rates are: one biopsy each at year one (81\%), year four (60\%), year seven (53\%) and year 10 (33\%). After year 10 biopsies are conducted every five years. An exception is made, if at any time a patient has PSA-DT less than 10 years, wherein the annual schedule is prescribed. It is important to note that, unlike biopsies, measurement of PSA has a high compliance rate of 91\% in PRIAS. This is because PSA is easy to obtain and measuring PSA does not lead to any side effects. The use of PSA-DT in the scheduling process is justified, because it was found to be indicative of GR in PRIAS \citep{bokhorst2015compliance}. 

This paper is motivated by the need to reduce the burden of biopsies and most optimally find the onset of GR. To this end, we intend to create personalized schedules for biopsies which improve upon the PRIAS and annual schedule. That is, a different schedule for every patient utilizing his recorded information. Personalized schedules for screening have received much interest in the literature, especially in the medical decision making context. For diabetic retinopathy, cost optimized personalized schedules based on Markov models have been developed by \citet{bebu2017OptimalScreening}. For breast cancer, personalized mammography screening policy based on the prior screening history and personal risk characteristics of women, using partially observable Markov decision process (MDP) models have been proposed by \citet*{ayer2012or}. MDP models have also been used to develop personalized screening policies for cervical cancer \citep*{akhavan2017markov} and colorectal cancer \citep*{erenay2014optimizing}. Another type of model called joint model for time to event and longitudinal data \citep{tsiatis2004joint,rizopoulos2012joint} has also been used to create personalized schedules, albeit for longitudinal biomarkers \citep{drizopoulosPersScreening}. In context of prostate cancer, \citet{zhang2012optimization} have used partially observable MDP models to personalize the decision of (not) deferring a biopsy to the next checkup time during the screening process. The decision is based on the baseline characteristics as well as a discretized PSA level of the patient at the current check up time.

Our work differs from the above referenced work in certain aspects. Firstly, the schedules we propose in this paper, account for the latent between-patient heterogeneity. We achieve this using joint models, which are inherently patient-specific because they utilize random effects. Secondly, joint models allow a continuous time scale and utilize the entire history of PSA levels. Lastly, instead of making a binary decision of (not) deferring a biopsy to the next check up time, we schedule biopsies at a per patient optimal future time, utilizing the historical PSA measurements and repeat biopsy results of the patient up to the current check up time. To this end, using joint models we first obtain a full specification of the joint distribution of PSA levels and time of GR. We then use it to define a patient-specific posterior predictive distribution of the time of GR given the observed PSA measurements and previous biopsies. Using the general framework of Bayesian decision theory, we propose a set of loss functions which are minimized to find the optimal time of conducting a biopsy. These loss functions yield us two categories of personalized schedules, those based on expected time of GR and those based on the risk of GR. We also analyze an approach where the two types of schedules are combined. To compare the proposed personalized schedules with the PRIAS and annual schedule we conduct a simulation study, and then discuss various criteria for evaluating the efficacy of each schedule, and a method to choose the most suitable one.

The rest of the paper is organized as follows. Section \ref{sec : jm_framework} briefly covers the joint modeling framework. Section \ref{sec : pers_sched_approaches} details the personalized scheduling approaches we have proposed in this paper. In Section \ref{sec : choosing_schedule} we discuss criteria for evaluation of the efficacy of a schedule and the choice of the optimal schedule. In Section \ref{sec : pers_schedule_PRIAS} we demonstrate the personalized schedules by employing them for the patients from the PRIAS program. Lastly, in Section \ref{sec: simulation_study}, we present the results from a simulation study we conducted to compare personalized schedules with PRIAS and annual schedule.