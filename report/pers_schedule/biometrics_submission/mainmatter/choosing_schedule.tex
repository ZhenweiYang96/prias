% !TEX root =  ../pers_schedules.tex

\section{Evaluation of Schedules}
\label{sec : choosing_schedule}
Given a particular schedule $S$ of biopsies, our next goal is to evaluate the schedule and to compare it with other schedules. To this end, we first present the methods to evaluate the biopsy schedules and then discuss the choice of a schedule.

We evaluate a schedule $S$ using two criteria, namely the number of biopsies $N^S_j \geq 1$ a schedule conducts for the $j$-th patient to detect GR, and the offset $O^S_j \geq 0$ by which it overshoots the true GR time $T^*_j$. The offset $O^S_j$ is defined as $O^S_j = T^S_{j{N^S_j}} - T^*_j$, where $T^S_{j{N^S_j}} \geq T^*_j$ is the time at which GR is detected. Our interest lies in the joint distribution $p(N^S_j, O^S_j)$ of the number of biopsies and the offset. Given the medical burden of biopsies, ideally only one biopsy with zero offset should be conducted. Hence, realistically we should select a schedule with a low mean number of biopsies $E(N^S_j)$ as well a low mean offset $E(O^S_j)$. It is also desired that a schedule has low variance of the number of biopsies $\mbox{var}(N^S_j)$, as well as low variance of the offset $\mbox{var}(O^S_j)$, so that the schedule works similarly for most patients. 

\subsection{Choosing a Schedule}
Given the multiple criteria for evaluation of a schedule, the next step is to use them to select a schedule. Using principles from compound optimal designs \citep{lauter1976optimal} we propose to choose a schedule $S$ which minimizes a loss function of the following form:
\begin{equation}
\label{eq : loss_func_sim_study_generic}
L(S) = \sum_{r=1}^R \eta_r \mathcal{R}_r(N^S_j).
\end{equation}
where $\mathcal{R}_r(\cdot)$ is an evaluation criteria based on either the number of biopsies or the offset (for brevity of notation, only $N^S_j$ is used in the equation above). Some examples of $\mathcal{R}_r(\cdot)$ are mean, median, variance and quantile function. Constants $\eta_1, \ldots, \eta_R$, where $0 \leq \eta_r \leq 1$ and $\sum_{r=1}^R \eta_r = 1$, are weights to differentially weigh-in the contribution of each of the $R$ criteria. An example loss function is:
\begin{equation}
\label{eq : loss_func_sim_study}
L(S) = \eta_1 E(N^S_j) + \eta_2 E(O^S_j). 
\end{equation}
The choice of $\eta_1$ and $\eta_2$ is not easy, because biopsies have associated medical risks and consequently the cost of an extra biopsy cannot be quantified or compared to a unit increase in offset easily. To obviate this problem we utilize the equivalence between compound and constrained optimal designs \citep{cook1994equivalence}. More specifically, it can be shown that for any $\eta_1$ and $\eta_2$ there exists a constant $C>0$ for which minimization of loss function in (\ref{eq : loss_func_sim_study}) is equivalent to minimization of the loss function subject to the constraint that $E(N^S_j) < C$. That is, a schedule which detects GR earliest, while simultaneously conducting at most $C$ biopsies on average, should be chosen. The choice of $C$ could be based on the number of biopsies a patient is willing to undergo. In the more generic case in (\ref{eq : loss_func_sim_study_generic}), a schedule can be chosen by minimizing $\mathcal{R}_R(\cdot)$ under the constraint $\mathcal{R}_r(\cdot) < C_r; r=1, \ldots, R-1$.