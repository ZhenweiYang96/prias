% !TEX root =  ../../pers_schedules.tex 

\subsection{Estimation}
Since there is no closed form solution available for $E_g(T^*_j)$, for its estimation we utilize the following relationship between $E_g(T^*_j)$ and $\pi_j(u \mid t, s)$:
\begin{equation}
\label{eq : expected_time_survprob}
E_g(T^*_j) = t + \int_t^\infty \pi_j(u \mid t, s) \rmn{d} u.
\end{equation}
There is no closed form solution available for the integral in (\ref{eq : expected_time_survprob}), and hence we approximate it using Gauss-Kronrod quadrature. We preferred this approach over Monte Carlo methods to estimate $E_g(T^*_j)$ from $g(T^*_j)$, because sampling directly from $g(T^*_j)$ involved an additional step of sampling from the distribution $p(T^*_j \mid T^*_j > t, \boldsymbol{b_j}, \boldsymbol{\theta})$, as compared to the estimation of $\pi_j(u \mid t, s)$ \citep{rizopoulos2011dynamic}. The former approach was thus computationally faster. 

As mentioned earlier, selection of the optimal biopsy time based on $E_g(T_j^*)$ alone will not be practically useful when the $\mbox{var}_g(T^*_j)$ is large, which is given by:
\begin{equation}
\label{eq : var_time_survprob}
\mbox{var}_g(T^*_j) = 2 \int_t^\infty {(u-t) \pi_j(u \mid t, s) \rmn{d} u} - \Big\{\int_t^\infty \pi_j(u \mid t, s) \rmn{d} u\Big\}^2.
\end{equation}
Since a closed form solution is not available for the variance expression, it is estimated similar to the estimation of $E_g(T^*_j)$. The variance depends both on last biopsy time $t$ and PSA history $\mathcal{Y}_j(s)$. The impact of the observed information on variance is demonstrated in Section \ref{subsec : demo_prias_pers_schedule}.

For schedules based on dynamic risk of GR, the value of $\kappa$ dictates the biopsy schedule and thus its choice has important consequences. In certain cases it may be chosen on the basis of doctor's advice or the amount of risk that is acceptable to the patient. For example, if the maximum acceptable risk is 5\%, then $\kappa = 0.95$.

In cases where $\kappa$ cannot be chosen on the basis of the input of the physician or the patients, we propose to automate the choice of $\kappa$. More specifically, we propose to choose a threshold $\kappa$ for which a binary classification accuracy measure \citep{lopez2014optimalcutpoints}, discriminating between cases and controls, is maximized. In PRIAS, cases are patients who experience GR and the rest are controls. However, a patient can be in control group at some time $t$ and in the cases at some future time point $t + \Delta t$, and thus time dependent binary classification is more relevant. In joint models, a patient $j$ is predicted to be a case if $\pi_j(t + \Delta t \mid t,s) \leq \kappa$ and a control if $\pi_j(t + \Delta t \mid t,s) > \kappa$ \citep*{rizopoulosJMbayes, landmarking2017}. The time window $\Delta t$ can be automatically chosen as $\argmax_{\Delta t}\mbox{AUC}(t, \Delta t, s)$, the latter being a measure of discriminative capability of the model \citep{rizopoulosJMbayes,landmarking2017}. However such a time window may not be clinically relevant at all. In AS programs at any point in time, it is of interest to identify patients who may obtain GR in the next one year from those who do not, so that they can be provided immediate attention (only in exceptional cases a biopsy within an year of the last one). Thus, in this work we use a $\Delta t$ of one year. As for the choice of the binary classification accuracy measure, we require a measure which is in line with the goal to focus on patients whose true time of GR falls in the time window $\Delta t$. To this end, the measure which combines both sensitivity and precision is $\mbox{F}_1$ score. It is defined as:
\begin{align*}
\mbox{F}_1(t, \Delta t, s) &= 2\frac{\mbox{TPR}(t, \Delta t, s) \mbox{PPV}(t, \Delta t, s)}{\mbox{TPR}(t, \Delta t, s) + \mbox{PPV}(t, \Delta t, s)}, \quad \mbox{F}_1 \epsilon [0,1],\\
\mbox{TPR}(t, \Delta t, s) &= \mbox{Pr}\big\{\pi_j(t + \Delta t \mid t,s) \leq \kappa \mid T^*_j \epsilon (t, t + \Delta t]\big\},\\
\mbox{PPV}(t, \Delta t, s) &= \mbox{Pr}\big\{T^*_j \epsilon (t, t + \Delta t] \mid \pi_j(t + \Delta t \mid t,s) \leq \kappa \big\}.
\end{align*}
where $\mbox{TPR}(\cdot)$ and $\mbox{PPV}(\cdot)$ denote time dependent true positive rate (sensitivity) and positive predictive value (precision), respectively. The estimation for both is similar to the estimation of $\mbox{AUC}(t, \Delta t, s)$ given by \citet{landmarking2017}. Since a high $\mbox{F}_1$ score is desired, the optimal value of $\kappa$ is $\argmax_{\kappa} \mbox{F}_1(t, \Delta t, s)$. In this work we compute the latter using a grid search approach. That is, first $\mbox{F}_1$ is computed using the available dataset over a fine grid of $\kappa$ values in the interval $[0,1]$, and then $\kappa$ corresponding to the highest $\mbox{F}_1$ is chosen.