% !TEX root =  ../pers_schedules.tex 
\section{Joint Model for Time to Event and Longitudinal Outcomes}
\label{sec : jm_framework}
We start with the definition of the joint modeling framework that will be used to fit a model to the available dataset, and then to plan biopsies for future patients. Let $T_i^*$ denote the true GR time for the $i$-th patient enrolled in an AS program. Let $S$ be the schedule of biopsies prescribed to this patient. The corresponding vector of time of biopsies is denoted by $T_i^S = \{T^S_{i0}, T^S_{i1}, \ldots, T^S_{i{N_i^S}}; T^S_{ij} < T^S_{ik}, \forall j<k\}$, where $N_i^S$ are the total number of biopsies conducted. Because of the periodical nature of biopsy schedules, $T_i^*$ cannot be observed directly and it is only known to fall in an interval $(l_i, r_i]$, where $l_i = T^S_{i{N_i^S - 1}}, r_i = T^S_{i{N_i^S}}$ if GR is observed, and $l_i = T^S_{i{N_i^S}}, r_i=\infty$ if GR is not observed yet. Further let $\bmath{y}_i$ denote the $n_i \times 1$  vector of PSA levels for the $i$-th patient. For a sample of $n$ patients the observed data is denoted by $\mathcal{D}_n = \{l_i, r_i, \bmath{y}_i; i = 1, \ldots, n\}$.

The longitudinal outcome of interest, namely PSA level, is continuous in nature and thus to model it the joint model utilizes a linear mixed effects model (LMM) of the form:
\begin{equation*}
\begin{split}
y_i(t) &= m_i(t) + \varepsilon_i(t)\\
&=\bmath{x}_i^T(t) \bmath{\beta} + \bmath{z}_i^T(t) \bmath{b}_i + \varepsilon_i(t),
\end{split}
\end{equation*}
where $\bmath{x}_i(t)$ denotes the row vector of the design matrix for fixed effects and $\bmath{z}_i(t)$ denotes the same for random effects. Correspondingly the fixed effects are denoted by $\bmath{\beta}$ and random effects by $\bmath{b}_i$. The random effects are assumed to be normally distributed with mean zero and $q \times q$ covariance matrix $\bmath{D}$. The true and unobserved PSA level at time $t$ is denoted by $m_i(t)$. Unlike $y_i(t)$, the former is not contaminated with the measurement error $\varepsilon_i(t)$. The error is assumed to be normally distributed with mean zero and variance $\sigma^2$, and is independent of the random effects $\bmath{b}_i$.

To model the effect of PSA on hazard of GR, joint models utilize a relative risk sub-model. The hazard of GR for patient $i$ at any time point $t$, denoted by $h_i(t)$, depends on a function of subject specific linear predictor $m_i(t)$ and/or the random effects:
\begin{align*}
h_i(t \mid \mathcal{M}_i(t), \bmath{w}_i) &= \lim_{\Delta t \to 0} \frac{\mbox{Pr}\big\{T^*_i \epsilon [t, t + \Delta t) \mid T^*_i \geq t, \mathcal{M}_i(t), \bmath{w}_i\big\}}{\Delta t}\\
&=h_0(t) \exp\big[\bmath{\gamma}^T\bmath{w}_i + f\{M_i(t), \bmath{b}_i, \bmath{\alpha}\}\big], \quad t>0,
\end{align*}
where $\mathcal{M}_i(t) = \{m_i(v), 0\leq v \leq t\}$ denotes the history of the underlying PSA levels up to time $t$. The vector of baseline covariates is denoted by $\bmath{w}_i$, and $\bmath{\gamma}$ are the corresponding parameters. The function $f(\cdot)$ parametrized by vector $\bmath{\alpha}$ specifies the functional form of PSA levels \citep{brown2009assessing,rizopoulos2012joint,taylor2013real,rizopoulos2014bma} that is used in the linear predictor of the relative risk model. Some functional forms relevant to the problem at hand are the following: 
\begin{eqnarray*}
\left \{
\begin{array}{l}
f\{M_i(t), \bmath{b}_i, \bmath{\alpha}\} = \alpha m_i(t),\\
f\{M_i(t), \bmath{b}_i, \bmath{\alpha}\} = \alpha_1 m_i(t) + \alpha_2 m'_i(t),\quad \text{with}\  m'_i(t) = \frac{\rmn{d}{m_i(t)}}{\rmn{d}{t}}.\\
\end{array}
\right.
\end{eqnarray*}
These formulations of $f(\cdot)$ postulate that the hazard of GR at time $t$ may be associated with the underlying level $m_i(t)$ of the PSA at $t$, or with both the level and velocity $m'_i(t)$ of the PSA at $t$. Lastly, $h_0(t)$ is the baseline hazard at time $t$, and is modeled flexibly using P-splines. More specifically:
\begin{equation*}
\log{h_0(t)} = \gamma_{h_0,0} + \sum_{q=1}^Q \gamma_{h_0,q} B_q(t, \bmath{v}),
\end{equation*}
where $B_q(t, \bmath{v})$ denotes the $q$-th basis function of a B-spline with knots $\bmath{v} = v_1, \ldots, v_Q$ and vector of spline coefficients $\gamma_{h_0}$. To avoid choosing the number and position of knots in the spline, a relatively high number of knots (e.g., 15 to 20) are chosen and the corresponding B-spline regression coefficients $\gamma_{h_0}$ are penalized using a differences penalty \citep{eilers1996flexible}. 

For the estimation of joint model's parameters we use a Bayesian approach. The details of the estimation method are presented in Web Appendix A of the supplementary material.