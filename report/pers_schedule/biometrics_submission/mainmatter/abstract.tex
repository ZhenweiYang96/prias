% !TEX root =  ../main.tex 

\begin{abstract}
Low risk prostate cancer patients are often encouraged to join active surveillance (AS) programs rather than taking immediate treatment. In most AS programs repeat biopsies are conducted annually or as per a common fixed schedule for these patients. When the Gleason score based on biopsies is found to be upgraded, the patients are given curative treatment. It has been found that such fixed and frequent schedules for biopsies discourage patients to receive repeat biopsies, and also bring a financial burden. Motivated by the world's largest AS program PRIAS, in this work we present personalized schedules for biopsies to counter these problems. Our methods create separate schedules for every person, on the basis of the evolution of his prostate-specific antigen (PSA) levels as well as results from previous repeat biopsies. We discuss criteria for evaluation of biopsy schedules, and then use them to compare the efficacy of personalized schedules with that of existing biopsy schedules.
\end{abstract}